\documentclass[11pt]{article}

\oddsidemargin=0.25truein \evensidemargin=0.25truein
\topmargin=-0.5truein \textwidth=6.0truein \textheight=8.75truein

%\RequirePackage{graphicx}
\usepackage{comment}
\usepackage{hyperref}
\urlstyle{rm}   % change fonts for url's (from Chad Jones)
\hypersetup{
    colorlinks=true,        % kills boxes
    allcolors=blue,
    pdfsubject={Data Bootcamp @ NYU Stern School of Business},
    pdfauthor={Dave Backus db3@nyu.edu},
    pdfstartview={FitH},
    pdfpagemode={UseNone},
%    pdfnewwindow=true,      % links in new window
%    linkcolor=blue,         % color of internal links
%    citecolor=blue,         % color of links to bibliography
%    filecolor=blue,         % color of file links
%    urlcolor=blue           % color of external links
% see:  http://www.tug.org/applications/hyperref/manual.html
}

%\renewcommand{\thefootnote}{\fnsymbol{footnote}}

% table layout
\usepackage{booktabs}

% section spacing and fonts
\usepackage[small,compact]{titlesec}

% list spacing
\usepackage{enumitem}
\setitemize{leftmargin=*, topsep=0pt}
\setenumerate{leftmargin=*, topsep=0pt, partopsep=0pt}

% attach files to the pdf
\usepackage{attachfile}
    \attachfilesetup{color=0.75 0 0.75}

\usepackage{needspace}
% example:  \needspace{4\baselineskip} makes sure we have four lines available before pagebreak

\usepackage{verbatim}

% change spacing of verbatim (not clear this does anything)
% http://tex.stackexchange.com/questions/43331/control-vertical-space-before-and-after-verbatim-environment
\usepackage{etoolbox}
\makeatletter
\preto{\@verbatim}{\topsep=0pt \partopsep=0pt}
\makeatother

% Make single quotes print properly in verbatim
\makeatletter
\let \@sverbatim \@verbatim
\def \@verbatim {\@sverbatim \verbatimplus}
{\catcode`'=13 \gdef \verbatimplus{\catcode`'=13 \chardef '=13 }}
\makeatother


% document starts here
\begin{document}
\parskip=\bigskipamount
\parindent=0.0in
\thispagestyle{empty}
{\large Data Bootcamp @ NYU Stern \hfill Dave Backus \& Glenn Okun}


\bigskip\bigskip
\centerline{\Large \bf Data Bootcamp:  Project Examples}
\centerline{Revised: \today}

\medskip
If you're having trouble finding a project idea, here are some you could use or adapt.

\section{Prepackaged projects}

{\bf Business conditions.\/}  Describe current business conditions in the US.
(Most banks produce this kind of thing, you should be able to find examples.)
This might include:
\begin{itemize}
\item A list of indicators and why you think they're useful.
\item An assessment of the indicators:  either the correlation with industrial production
or some other quantitative measure of how closely it tracks the economy.
\item A summary of what your indicators overall suggest about the current state
of the US economy.
\end{itemize}
Chapter 11 of the \href{(http://www.stern.nyu.edu/experience-stern/about/departments-centers-initiatives/centers-of-research/global-economy-business/development-initiatives/global-economy-course}
{Global Economy book} has a nice overview of what's involved
and FRED codes for some popular indicators.

{\bf Emerging market opportunities.\/}


{\bf IPython notebook of data input chapter.\/}
Create a notebook that goes through the material of the data input chapter of the book,
or perhaps one of the other early chapters.
The result should be a notebook that allows others to teach themselves.
Include blank cells for exercises.


\section{More work..}

{\it Demography in Japan.\/}

Links to pop and projections, fertility rates...


\section*{Datasets worth developing}



{\bf Business conditions.\/}


{\vfill
{\bigskip \centerline{\it \copyright \ \number\year \
David Backus $|$ NYU Stern School of Business}%
}}


\end{document}
