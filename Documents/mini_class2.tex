\documentclass[11pt]{article}

\oddsidemargin=0.25truein \evensidemargin=0.25truein
\topmargin=-0.5truein \textwidth=6.0truein \textheight=8.75truein

%\RequirePackage{graphicx}
\usepackage{comment}
\usepackage{hyperref}
\urlstyle{rm}   % change fonts for url's (from Chad Jones)
\hypersetup{
    colorlinks=true,        % kills boxes
    allcolors=blue,
    pdfsubject={Data Bootcamp @ NYU Stern School of Business},
    pdfauthor={Dave Backus db3@nyu.edu},
    pdfstartview={FitH},
    pdfpagemode={UseNone},
%    pdfnewwindow=true,      % links in new window
%    linkcolor=blue,         % color of internal links
%    citecolor=blue,         % color of links to bibliography
%    filecolor=blue,         % color of file links
%    urlcolor=blue           % color of external links
% see:  http://www.tug.org/applications/hyperref/manual.html
}

%\renewcommand{\thefootnote}{\fnsymbol{footnote}}

% table layout
\usepackage{booktabs}

% section spacing and fonts
\usepackage[small,compact]{titlesec}

% list spacing
\usepackage{enumitem}
\setitemize{leftmargin=*, topsep=0pt}
\setenumerate{leftmargin=*, topsep=0pt}

% attach files to the pdf
\usepackage{attachfile}
    \attachfilesetup{color=0.75 0 0.75}

\usepackage{needspace}
% example:  \needspace{4\baselineskip} makes sure we have four lines available before pagebreak

\usepackage{verbatim}


% Make single quotes print properly in verbatim
\makeatletter
\let \@sverbatim \@verbatim
\def \@verbatim {\@sverbatim \verbatimplus}
{\catcode`'=13 \gdef \verbatimplus{\catcode`'=13 \chardef '=13 }}
\makeatother


% document starts here
\begin{document}
\parskip=\bigskipamount
\parindent=0.0in
\thispagestyle{empty}
{\large Data Bootcamp @ NYU Stern \hfill Dave Backus \& Glenn Okun}


\bigskip\bigskip
\centerline{\Large \bf Data Bootcamp:  Class \#2}
\centerline{Revised: \today}


\begin{comment}
* Today:  graphics
* Something to do:  look for cool graphics, think about how you'd do it, also data
* Warning:  need to work.  Introduce yourself to your neighbor.  Raise your hand if you have trouble.
* Before we start
    - start Spyder
    - download bootcamp_graphics.py
\end{comment}


\section{Course overview}

\begin{itemize}
\item Objective:  Learn enough about Python to do useful things with data.
\item Target audience:  Programming newbies.
Anyone can do this with a little persistence and the help of friends.
\item Trigger warning:  This will take some effort.  But it's worth it.

\item {\bf Work habits:}
\begin{itemize}
\item Advice.  This will seem mysterious at first, but if you stick with it,
you'll find it starts to look familiar, even make sense.
\item Practice.  Any time you have something to do with data, try it out in Python.
Play around.  Have fun!
\item Friends.  Work with friends, and make new friends who know how to code.
\item Help.  If you get stuck, ask for help --- from friends,
from the Bootcamp group (post a problem), Stack Overflow, etc.
\end{itemize}
\item Team:  Dave Backus, Glenn Okun, Sarah Beckett-Hile, and a rotating group of ninjas.

\item Resources:
\begin{itemize}
\item Bootcamp Group:  \url{https://groups.google.com/forum/#!forum/nyu_data_bootcamp}.
Post comments and questions here.

\item GitHub repository:  \url{https://github.com/DaveBackus/Data_Bootcamp}.
All the docs and programs are here.
This document is in the Notes folder;
the pdf file comes with links.
Programs are in the Code folder under Python.
\end{itemize}

\end{itemize}


\section{Overall plan}

\begin{itemize}
\item First class:  Python basics, examples.
\item Today's class:  graphics.
\item Next (and last class) in two weeks:  data management, SQL
\item {\bf It's possible we'll run a more formal course in August.}
If we do, there will be greater emphasis on data.
\end{itemize}


\section{Today's plan}

\begin{itemize}
\item Theme:  graphics.
\item Standard ``scientific'' packages.
\item Reminders.
\item Graphics with Matplotlib: two routes, one destination.
\item Exercise review (Sarah).
\end{itemize}


\section{Standard packages}

Python is not just a program, it's an open source collection of tools
that includes both basic
Python and a collection of packages written by different people.
The good news is that you can often find a package to do whatever it is
you want to do.
The bad news is that there's not necessarily any uniformity or quality control.
The standard packages --- which we'll talk more about in a minute ---
are well written, well documented, and have armies of users who spot and correct
problems.
Some of the others less so.
Our suggestion is to stick to the mainstream packages, such as those in the
\href{http://docs.continuum.io/anaconda/pkg-docs.html}{Anaconda distribution},
until you're reasonably comfortable with Python.


Some of the leading mainstream packages for numerical (``scientific'') computation ---
 all of which are in the Anaconda --- are
\begin{itemize}
\item \href{http://www.numpy.org/}{NumPy}.
Tools for numerical computing, including linear algebra.
Compare this to Excel, where the basic unit is a cell, a single number.
In NumPy the basic unit is a vector (a column) or matrix (a table).
In this respect, it's a direct competitor to Matlab.
We won't see much of NumPy here, but it's the foundation for Pandas.

\item \href{http://matplotlib.org/index.html}{Matplotlib}.
The leading graphics library for Python and our focus today.

\item \href{http://pandas.pydata.org/}{Pandas}.
Tools for managing and manipulating data.
The basic unit is a DataFrame, which is a table of data with labels
for observations and variables.
If you've used R, you'll be familiar with the idea.
If not, you'll be familiar with it in a couple weeks.
We'll touch on this today and spend a fair amount of time with it at
our next meeting.
\end{itemize}

We love the term ``wrapper,'' it makes it sound like we know what we're doing.
It's an informal programming term that means a simplified interface
for more basic programming functions. Cool, eh?
Here are some popular wrappers for Matplotlib:
\begin{itemize}
\item \href{http://matplotlib.org/users/pyplot_tutorial.html}{Pyplot}.
This has two interfaces for Matplotlib,
a simple one similar to Matlab and a more complicated object-oriented one
(which means we'll use objects and methods).
We'll describe both.

\item \href{http://web.stanford.edu/~mwaskom/software/seaborn/}{Seaborn}.
Another interface for Matplotlib.  Not part of the Anaconda distribution,
but it seems to us to have some potential.
Take a look at the examples, see what you think.

\item \href{http://wiki.scipy.org/PyLab}{PyLab}.
A combination of (basically) NumPy and Pyplot designed to replace Matlab.
Convenient if you're familiar with Matlab, probably not otherwise.
The
\href{http://matplotlib.org/dev/faq/usage_faq.html#matplotlib-pylab-and-pyplot-how-are-they-related}
{Matplotlib FAQ}
adds:  ``it is no longer recommended.''

\end{itemize}
We'll use Pyplot today.
And Sarah may use Seaborn when she goes through last week's exercise.

**** More:
\href{http://statsmodels.sourceforge.net/}{statsmodels} (statistics)
and
\href{http://scikit-learn.org/stable/}{scikit} (machine learning).
Both useful, and there's a fair amount of overlap.
% ??

** Natural language processing??


\section{Reminders}

\begin{itemize}
\item Environment:  Python 3.4 in Spyder.
\item Assignments, strings, lists, slicing:
\begin{verbatim}
x, y, z = 2*3, 2**3, 2/3    # yes, three at once!
a, b = 'some', 'thing'
c = a + b

numbers = [x, y, z]
strings = [a, b, c]
both = numbers + strings
print(both)
print([type(both), both[:3], both[3:])
\end{verbatim}

\item Objects and methods.  Most of the things we've seen are objects:  {\tt x, a, all}, etc.
They have methods defined for them -- useful things you can do with them.
For more on the methods available for the object {\tt all},
go to the IPython console and type {\tt all.[tab]}.
We call this ``tab completion.''
For more on a specific method called {\tt foo},
go to the Object explorer and type {\tt all.foo}.

\item Reading csv files.
This was part of last week's assignment:
\begin{verbatim}
import pandas as pd
url1 = 'https://raw.githubusercontent.com/fivethirtyeight/data/master/'
url2 = 'college-majors/recent-grads.csv'
df = pd.read_csv(url1+url2)
\end{verbatim}
What is {\tt df}?  What's in it?
\begin{verbatim}
properties = [type(df), df.columns, df.index, df.shape, df.head()]
for property in properties :
    print(property, end='\n\n')
\end{verbatim}
Use tab completion to find more.


\item {\bf Exercises.}
\begin{itemize}
\item How do we find the first item in the list {\tt all}?  The last?
\item What is {\tt len(all)}?
\item How do we find the third column name in {\tt df}?  What is it?
\end{itemize}

\end{itemize}


\section{Graphics overview}

Graphics are horrendously complicated.  Programs like Excel
do their best to hide this fact, but if you ever try to customize
a chart it quickly rears its ugly head.
A graph might include, for example:  the type (line, bar, etc);
the color and thickness of lines, bars, or markers;
title and axis labels;
their location, fonts, and font sizes;
tick marks (location, size);
background color;
and so on.

Matplotlib gives you control over all of these things and more ---
but it's complicated, there are lots of moving parts.
The set of defaults is collected in what they call the
{\tt matplotlibrc} file, and
\href{http://matplotlib.org/1.4.0/users/customizing.html}{it's enormous}.
That's why people have come up with simpler interfaces,
Pyplot and Seaborn among them.

Bottom line:  Graphics are complicated, but that's the nature of the beast,
not something specific to Python.
What Python gives you is total control.
As always, you should start simple, work up to more complicated things gradually.


\section{Pyplot's simple interface}

% ?? more exercises

\begin{itemize}
\item Overview.  Pyplot is a module of Matplotlib.
Here we'll use its simple interface.
It looks like Matlab, but if that means nothing to you forget we said it.
Either way, it's a good place to start.

\item References:
\begin{itemize}
\item Matplotlib's
\href{http://matplotlib.org/users/pyplot_tutorial.html}{Pyplot tutorial}.
(This is excellent, check it out.)
\item Matplotlib's
\href{http://matplotlib.org/gallery.html}{gallery of examples}.
Graphs along with the code that produced them, a great place to start.

\item  Rougier, Mueller, and Varoquaux's
\href{https://scipy-lectures.github.io/intro/matplotlib/matplotlib.html}{SciPy lecture notes}.
\item Sargent and Stachurski's ``Quantitative Economics,''
the section on
\href{http://quant-econ.net/py/matplotlib.html}{Matplotlib}.
Good, but a little terse for our taste.
\item \href{http://pbpython.com/visualization-tools-1.html}{Overview} of Python visualization packages.
\end{itemize}

\item Line plots.
Here we take 11 years of annual US GDP data and plot it against time.
Also consumption (``personal consumption expenditures''), so that we can
show how to add a second line to a graph.
We put all of these things, plus time, in lists and go to work:
\begin{verbatim}
import matplotlib.pyplot as plt     # import pyplot module

# data
gdp  = [13271.1, 13773.5, 14234.2, 14613.8, 14873.7, 14830.4, 14418.7,
        14783.8, 15020.6, 15369.2, 15710.3]
pce  = [8867.6, 9208.2, 9531.8, 9821.7, 10041.6, 10007.2, 9847.0, 10036.3,
        10263.5, 10449.7, 10699.7]
year = [2003, 2004, 2005, 2006, 2007, 2008, 2009, 2010, 2011, 2012, 2013]

# plot
plt.plot(year, gdp)         #  plot(x,y)
\end{verbatim}
We can do lots of things to dress up the plot.
Here are some examples, but find your own with {\tt plt.[tab]}:
\begin{verbatim}
plt.plot(year, pce, color='magenta', linewidth=2)  # add another line
plt.title('US Real GDP and Consumption')    # title
plt.ylabel('Billions of 2009 USD')          # y axis label
plt.legend(('GDP', 'Consumption'), loc=0)   # legend
plt.text(2010, 14000, 'GDP')                # add text at specific location
plt.text(2010, 9200, 'Consumption')         # same again
plt.ylim((0, 16000))                        # change y axis limits
plt.tick_params(length=6, width=2, colors='red')    # yikes!
plt.savefig('bootcamp_test.pdf')            # export as (in this case) pdf
plt.show()                                  # close plot
\end{verbatim}

\item {\bf Exercise.} Run the basic plot GDP on your computer and
\begin{itemize}
\item Change the color of the GDP line to green.
\item Add a marker.
\item Left-justify the title.
\item Extra credit.  Change the title's font size to 20.
\item More extra credit.  What does the {\tt plot} argument {\tt alpha=0.5} do?
\end{itemize}


\item Bar charts.
We'll work with two examples here.
The first is to express our line plot of US GDP as a bar chart.
We do that with the command:
\begin{verbatim}
plt.bar(year, gdp)                          # bar(x,y)
\end{verbatim}
This isn't the most attractive bar chart in the world, but it works.
Here's an example that looks better:
\begin{verbatim}
plt.bar(year, gdp, width=0.9, color='blue', align='center', alpha=0.7)
plt.show()
\end{verbatim}


Here's another example, a standard chart in the MBA Global Economy class.
The difference here is that one axis is a country, which is not numerical data.
(There's a similar
\href{http://matplotlib.org/examples/lines_bars_and_markers/barh_demo.html}{example}
in the Matplotlib gallery.)
Here's the data:
\begin{verbatim}
codes     = ['USA', 'FRA', 'JPN', 'CHN', 'IND', 'BRA', 'MEX']
countries = ['United States', 'France', 'Japan', 'China', 'India',
             'Brazil', 'Mexico']
gdppc     = [53.1, 36.9, 36.3, 11.9, 5.4, 15.0, 16.5]
\end{verbatim}
Now think about constructing a bar chart.
The Pyplot command is {\tt bar},
which works just like {\tt plot}:  the first two arguments
are the x and y axes.
Here there's no obvious x axis --- the country codes and names won't work ---
so we need to make one.
These two lines (i)~make up a counter of the same length as {\tt gdppc}
that we can use
and (ii)~generate a bar chart.
\begin{verbatim}
other_axis = range(len(gdppc))
print(other_axis)
plt.bar(other_axis, gdppc, align='center')
\end{verbatim}
This isn't the most elegant chart you'll see, but it's functional.

You might stop here and think:  What would you do to dress this chart up,
make it more attractive and compelling?

Here are some examples that might get you started:
\begin{verbatim}
plt.xticks(other_axis, codes, fontsize=12)
plt.xlim((-0.75, 6.75))
plt.ylabel('GDP Per Capita (thousands of USD)')
plt.title('GDP Per Capita', fontsize=16, loc='left')
\end{verbatim}
What do you think?  What would you add?

\item Horizontal bar charts.
The same chart looks better if we rotate it.
That's easily done by replacing the command {\tt bar} with {\tt barh}.
Even better, we now have room to add the country names, not just the three-letter
country codes.
\begin{verbatim}
plt.barh(other_axis, gdppc, align='center', alpha=0.7)
plt.ylim((-0.6, 6.6))
plt.yticks(other_axis, countries, fontsize=14)
\end{verbatim}

\item *** Titanic example, survival by class.  
\url{http://www.analyticsvidhya.com/blog/2014/08/baby-steps-python-performing-exploratory-analysis-python/} 

\end{itemize}



\section{Pyplot's object-oriented interface}

\begin{itemize}
\item Overview.
This is close to classic Matplotlib,
which is said to give you greater flexibility and control.
You'll have to decide for yourself whether it's worth it.
Eventually yes, but it probably makes sense to start with simple interface.
We'll go through this at high speed to give you a sense
of the syntax.

\item References:
\begin{itemize}
\item \href{http://matplotlib.org/index.html}{Matplotlib}
doesn't have a tutorial.
The documentation sends you to the gallery of examples and outside sources.
\item Sargent and Stachurski's ``Quantitative Economics'' is pretty good, if a little terse.
See the section on
\href{http://quant-econ.net/py/matplotlib.html}{Matplotlib}.
\end{itemize}

\item Basic syntax.  A typical graph has commands like these:
\begin{verbatim}
import matplotlib.pyplot as plt
fig, ax = plt.subplot()
\end{verbatim}
{\tt fig} and {\tt ax} here are variable names;
we can call them anything we want, but these are standard.
{\tt fig} is referred to as a figure object, it controls the overall figure area.
If we stop here, we just get a blank box.
{\tt ax} is referred to as an axis object,
 it controls the content of the figure.
 We'll give some examples below.

\item Line plots.  We'll start with the same data we used earlier:
\begin{verbatim}
gdp  = [13271.1, 13773.5, 14234.2, 14613.8, 14873.7, 14830.4, 14418.7,
        14783.8, 15020.6, 15369.2, 15710.3]
pce  = [8867.6, 9208.2, 9531.8, 9821.7, 10041.6, 10007.2, 9847.0, 10036.3,
        10263.5, 10449.7, 10699.7]
year = [2003, 2004, 2005, 2006, 2007, 2008, 2009, 2010, 2011, 2012, 2013]
\end{verbatim}
To generate a plot of US real GDP, we apply the method {\tt plot}
to {\tt ax} much as we did before:
\begin{verbatim}
fig, ax = plt.subplots()                    # sets up single plot
ax.plot(year, gdp, color='blue', linewidth=2, alpha=0.8)
plt.show()                                  # closes plot
\end{verbatim}

\item Multiple plots.
\begin{verbatim}
fig, ax = plt.subplots(nrows=2, ncols=1, sharex=True)
# first figure
ax[0].plot(year, gdp, color='blue', linewidth=2, alpha=0.8)
ax[0].text(2010, 14000, 'GDP')
# second figure
ax[1].plot(year, pce, color='magenta', linewidth=2, alpha=0.8)
ax[1].text(2010, 9200, 'Consumption')
plt.show()
\end{verbatim}

*** Do also the object-oriented version

\item {\bf Exercise.}
Convert the double line plot above to bar charts.


\end{itemize}



\section{Before the next class}

{\bf Exercise.}
Randy Olson has a
\href{http://www.randalolson.com/2014/06/28/how-to-make-beautiful-data-visualizations-in-python-with-matplotlib/}
{nice blog post} (search ``randy olson beautiful'')
about creating effective graphics.
He starts with this gif:
\url{http://gfycat.com/ImprobableFemaleBasenji}.
Skim his post, play the gif, and
%
\begin{itemize}
\item  Write a Python script that recreates the data in the figure,
both numbers and labels.
(There are only a few points, you can just type them into lists.)
\item Approximate the various graphs using Matplotlib.
\item Play around with the graphics parameters.
What is the best graph you can produce?
What do you like about it?
\item Variant:  Do the same with our GDP per capita bar charts.
\end{itemize}

{\bf Exercise.\/}
Do separate bar charts for GDP per capita in China, India, and the US:
one for 1980, another for 2014 (or the most recent available date).
\begin{itemize}
\item Do separate charts.
\item Put them in one two-part figure, 1980 at the top, 2014 at the bottom.
\item Combine them in one chart.  What configuration looks best to you?
\end{itemize}


\section*{Today's code}

Attached.  Download this pdf file, open in Adobe Acrobat or the equivalent,
and click on the pushpins:
\attachfile{../Code/Python/bootcamp_mini_graphics.py}

Also here:
{\small
\verbatiminput{../Code/Python/bootcamp_mini_graphics.py}
}

\end{document}


