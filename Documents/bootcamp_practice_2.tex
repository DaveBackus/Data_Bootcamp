\documentclass[11pt]{exam}

\oddsidemargin=0.25truein \evensidemargin=0.25truein
\topmargin=-0.5truein \textwidth=6.0truein \textheight=8.75truein

%\RequirePackage{graphicx}
\usepackage{comment}
\usepackage{hyperref}
\urlstyle{rm}   % change fonts for url's (from Chad Jones)
\hypersetup{
    colorlinks=true,        % kills boxes
    allcolors=blue,
    pdfsubject={Data Bootcamp @ NYU Stern School of Business},
    pdfauthor={Dave Backus db3@nyu.edu},
    pdfstartview={FitH},
    pdfpagemode={UseNone},
%    pdfnewwindow=true,      % links in new window
%    linkcolor=blue,         % color of internal links
%    citecolor=blue,         % color of links to bibliography
%    filecolor=blue,         % color of file links
%    urlcolor=blue           % color of external links
% see:  http://www.tug.org/applications/hyperref/manual.html
}

%\renewcommand{\thefootnote}{\fnsymbol{footnote}}

% table layout
\usepackage{booktabs}

% section spacing and fonts
\usepackage[small,compact]{titlesec}

% list spacing
\usepackage{enumitem}
\setitemize{leftmargin=*, topsep=0pt}
\setenumerate{leftmargin=*, topsep=0pt}

% attach files to the pdf
\usepackage{attachfile}
    \attachfilesetup{color=0.75 0 0.75}

\usepackage{needspace}
% example:  \needspace{4\baselineskip} makes sure we have four lines available before pagebreak

\usepackage{verbatim}


% Make single quotes print properly in verbatim
\makeatletter
\let \@sverbatim \@verbatim
\def \@verbatim {\@sverbatim \verbatimplus}
{\catcode`'=13 \gdef \verbatimplus{\catcode`'=13 \chardef '=13 }}
\makeatother


%\printanswers

% document starts here
\begin{document}
\parskip=\bigskipamount
\parindent=0.0in
\thispagestyle{empty}
{\large Data Bootcamp @ NYU Stern \hfill Dave Backus \& Glenn Okun}


\bigskip\bigskip
\centerline{\Large \bf Data Bootcamp:  Code Practice \#2}
\centerline{Revised: \today}

{\it Answer each of the questions below.
Answers should include any code you write.
They can be handwritten, but they must be readable and look professional.}

\begin{questions}

\item {\it Review.\/} Does this code run without error? If so, what does it produce?  If not, explain why.
\begin{verbatim}
x = [1, 2, 3]
y = 'bootcamp'
z = x + y
\end{verbatim}

\item {\it Review.\/} For the same {\tt x} and {\tt y} as the previous question:
What function tells us what type they are?
What function tells us how many elements they contain
(their ``length,'' so to speak)?

%\item {\it Review.\/} Does this code run without error? If so, what does it produce?  If not, explain why.
%\begin{verbatim}
%x = [1, 2, 3]
%y = [4]
%z = x + y
%\end{verbatim}

%\item (Review) What is the value of \texttt{y} after running the code below?
%\begin{verbatim}
%x = [1]
%y = x
%x.append(2)
%y = y.pop()
%\end{verbatim}

% Logical Questions
\item What type is each expression?  How can you tell?
\begin{verbatim}
2
'2'
2.0
"2.0"
2>1
'Itamar' > 'Chase'
[1, 2]
(1, 2)
{1: one, 2: two}
\end{verbatim}

\item What value does each of these comparisons have?
\begin{verbatim}
1>=0
1 >= 1
1 > 1
1==1
1 == 1.0
'Spencer' == "Spencer"
\end{verbatim}

\item  Does this code run without error?  If so, what does it produce?  If not, how would you fix it?
\begin{verbatim}
if 2>1
   print('Yes, 2 is still greater than 1')
\end{verbatim}
  % Answer:  missing :, indented 3

\needspace{2\baselineskip}
\item What is the result of running this code?  Why?
\begin{verbatim}
if True:
    print('on the one hand')
else:
    print('but on the other hand')
\end{verbatim}
What happens if we replace {\tt True} with {\tt False} in the first line?

\item What is the result of running this code?
\begin{verbatim}
cond = True
if cond:
    x = "Chase"
else:
    x = "Dave"
print(x)
\end{verbatim}

\item Suppose we have two lists,
{\tt x = [1, 2, 3, 4]} and \verb|y = ['x', 'y', 'z']|.
Adapt the code below to determine which has more elements:
\begin{verbatim}
if <insert expression>:
    print('x has more')
else:
    print('y has more')
\end{verbatim}

%\item What do you expect the output of each of the following comparisons to be?
%\begin{verbatim}
%print("sarah" == 'sarah')
%print([1, 2, 3] == [1, 2, 3])
%print([0, 1, 2] == [1, 2, 3])
%\end{verbatim}

%\item Write a short piece of code (using an if-else statement) that
%assigns a numerical value to a variable \texttt{x} and then checks whether the type of \texttt{x} is a float or integer. Print out which it is. ??

% Slicing
\item Explain in words what slicing does.

\item How would you extract (``slice'') the first element (the integer {\tt 1}) from the list {\tt x} below?
The last element?  All but the last element?
\begin{verbatim}
x = [1, 2, 3, 4, 5]
\end{verbatim}

\item Use slicing to extract each word from
\begin{verbatim}
sentence = 'This is a sentence; please slice it.'
\end{verbatim}
{\it Suggestion:\/} Number every character in {\tt sentence} by hand.


\item Consider the list
\begin{verbatim}
x = [1, 2, "a", 'b', "fast", 'slow', 3, "Raghu", 'Liuren', 10]
\end{verbatim}
\begin{parts}
\item How would you slice out the first item?  The last item?
\item How would you slice out the items from {\tt 'b'} to {\tt 3} inclusive?
\end{parts}

\item Using the same list {\tt x}, write a loop that prints every element on a new line.

\item {\it Challenging.\/}
Using the same list {\tt x}, write a loop that prints every element of type {\tt str}.

% Looping
%\item Finish writing the for loop below so that it adds all of the numbers less than 100.
%\begin{verbatim}
%x = 0
%for i in :
%    x += i
%    print(x)
%\end{verbatim}

\begin{comment} 
\item Suppose the current price of corn is \$1.20.
The price is set to increase by 10 cents every weekday until it hits \$2.50.
(Nothing happens on weekends.)
At that point it will switch directions and decrease by 10 cents a day. 
Finish the code below to print the price of corn every day until it reaches \$2.50.
\begin{verbatim}
days = ["M", "T", "W", "Th", "F"]
starting_price = 1.20
current_price = starting_price
while True:
for day in days:
    current_price = current_price + 0.10
    print("The price on ", day, "is ", current_price)
if current_price == 2.50:
    break
\end{verbatim}
\end{comment}


%% Functions
%\item Define a function that takes your birth year and the current year as inputs
%and prints every year in between (and including) those years.


%\item A function is defined below. Determine what it does and write an example of you using it.
%\begin{verbatim}
%def score_to_percentage(pointsawarded, pointstotal):
%    """
%    This function changes a number of points awarded
%    and a number of total points into a percentage
%    """
%    percentage = pointsawarded / pointstotal
%    return percentage
%\end{verbatim}

%\item {\it Challenging.\/}
%Create a function to convert a date tuple to a regular date:
%to convert (say) \texttt{(2015, 8, 22)} to \texttt{August 22, 2015} expressed as a string.
%That is:  The input is a tuple and the return is a string.
%{\it Suggestion:\/} You might want to use the list
%\begin{verbatim}
%months = ['January', 'February', etc]
%\end{verbatim}

\item Define a function \verb|pocket_change()| that takes four integers as inputs 
(numbers of pennies, nickels, dimes, and quarters in your pocket)
and returns a floating point number (their dollar value).  
Run your program with two sets of inputs.  

\item {\it Challenging.\/}
Write a function that takes a list of integers
and returns a (shorter) list of those that do not begin with 6.


\item Use Spyder's help to find out what the range function does.  
How would you describe {\tt range(3,12,2)}?  
Verify by converting to a list with {\tt list(range(3,12,2))}.

\item {\it Challenging.\/}
Write a loop that sums the integers from zero to thirty that are multiples of three:
3, 6, etc.
{\it Hints:}  You can create a blank list with {\tt x = []}.  
You can append an item to a list {\tt x} with {\tt x.append(item)}. 


\item Consider the Python object
\begin{verbatim}
z = {1: 'one', 2: 'two', 3: 'three'} 
\end{verbatim} 
\begin{parts}
\item What kind of object is {\tt z}?  What is its length?  
\item Which components are keys?  Which are values? 
\item How would I get the value associated with the key 2?
\item Use Spyder's help facilities to figure out what {\tt z.keys()} does.  
Ditto {\tt z.values()}.  Try them to verify.  
\item What does {\tt list(z.keys())} do?
\item What does {\tt list(z.values())} do?
\item What does {\tt list(z)} do?
\end{parts} 


\item {\it Challenging.\/} 
Take the dictionary connecting state names and abbreviations:
\begin{verbatim}
states = {'AK': 'Alaska', 'IA': 'Iowa', 'NY': 'New York', 'VT': 'Vermont'} 
\end{verbatim} 
Write a function that takes a string as input (the abbreviation) 
and returns another string (the full name of the state).
Test your function on two test cases.  


\end{questions}


\end{document}
