\documentclass[11pt]{exam}

\oddsidemargin=0.25truein \evensidemargin=0.25truein
\topmargin=-0.5truein \textwidth=6.0truein \textheight=8.75truein

%\RequirePackage{graphicx}
\usepackage{comment}
\usepackage{hyperref}
\urlstyle{rm}   % change fonts for url's (from Chad Jones)
\hypersetup{
    colorlinks=true,        % kills boxes
    allcolors=blue,
    pdfsubject={Data Bootcamp @ NYU Stern School of Business},
    pdfauthor={Dave Backus db3@nyu.edu},
    pdfstartview={FitH},
    pdfpagemode={UseNone},
%    pdfnewwindow=true,      % links in new window
%    linkcolor=blue,         % color of internal links
%    citecolor=blue,         % color of links to bibliography
%    filecolor=blue,         % color of file links
%    urlcolor=blue           % color of external links
% see:  http://www.tug.org/applications/hyperref/manual.html
}

%\renewcommand{\thefootnote}{\fnsymbol{footnote}}

% table layout
\usepackage{booktabs}

% section spacing and fonts
\usepackage[small,compact]{titlesec}

% list spacing
\usepackage{enumitem}
\setitemize{leftmargin=*, topsep=0pt}
\setenumerate{leftmargin=*, topsep=0pt}

% attach files to the pdf
\usepackage{attachfile}
    \attachfilesetup{color=0.75 0 0.75}

\usepackage{needspace}
% example:  \needspace{4\baselineskip} makes sure we have four lines available before pagebreak

\usepackage{verbatim}


% Make single quotes print properly in verbatim
\makeatletter
\let \@sverbatim \@verbatim
\def \@verbatim {\@sverbatim \verbatimplus}
{\catcode`'=13 \gdef \verbatimplus{\catcode`'=13 \chardef '=13 }}
\makeatother


% document starts here
\begin{document}
\parskip=\bigskipamount
\parindent=0.0in
\thispagestyle{empty}
{\large Data Bootcamp @ NYU Stern \hfill Dave Backus \& Glenn Okun}


\bigskip\bigskip
\centerline{\Large \bf Data Bootcamp:  Code Practice \#2}
\centerline{Revised: \today}

{\it Answer each of the questions below.
Answers should include any code you wrote to find your answer.
They can be handwritten, but they must be readable and look professional.}

\begin{questions}
\item {\it Review.\/}
Does this code run without error?  If so, what does it produce?  If not, explain why.
\begin{verbatim}
x = [1, 2, 3]
y = [42]
z = x + y
print(z)
\end{verbatim}

\item {\it Review.\/}
Does this code run without error?  If so, what does it produce?  If not, explain why.
\begin{verbatim}
x = "I don't think that word means"
y = 'what you think it means'
z = x + y
print(z)
\end{verbatim}


\item Describe and explain the type of these expressions:
\begin{verbatim}
12
12.0
'12.0'
"""12.0"""
[12]
1 >= 2
1<=2
'12.0' == """12.0"""
\end{verbatim}


\item Describe and explain the result of each of these comparisons:
\begin{verbatim}
2 >= 1
2 == 1
'12' == "12"
[1] + [2] == [1, 2]
'Ruinan' <= 'Zheng'
\end{verbatim}


\item If statements


\item
Does this code run without error?  If so, what does it produce?  If not, how would you fix it?
\begin{verbatim}
if 2>1
   print('Yes, 2 is still greater than 1')
\end{verbatim}
% Answer:  missing :, indented 3


\item Slicing strings and lists...


\item Loops over strings and lists

\item Loops over range


\item Function defs


\item Dictionaries and tuples

\item
Does this code run without error?  If so, what does it produce?  If not, explain why.
\begin{verbatim}
x = (1, 2, 3)
y = (42, 43)
z = x + y
\end{verbatim}



\item {\it Challenging.\/}
Create a function to convert a date tuple to a regular date:
to convert (say) \texttt{(2015, 8, 22)} to \texttt{August 22, 2015} expressed as a string.
That is:  The input is a tuple and the return is a string.
{\it Suggestion:\/} You might want to use the list
\begin{verbatim}
months = ['January', 'February', etc]
\end{verbatim}

\item Convert \texttt{range(4)} to a list.


\item Function:  take list of change [quarters, dimes, nickels, pennies] and return value. 
{\it Bonus points:\/} Use a tuple.  

\item Function:  take exam score, return a, b, c.

\end{questions}


\end{document}
