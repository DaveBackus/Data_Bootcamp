\documentclass[11pt]{article}

\oddsidemargin=0.25truein \evensidemargin=0.25truein
\topmargin=-0.5truein \textwidth=6.0truein \textheight=8.75truein

%\RequirePackage{graphicx}
\usepackage{comment}
\usepackage{hyperref}
\urlstyle{rm}   % change fonts for url's (from Chad Jones)
\hypersetup{
    colorlinks=true,        % kills boxes
    allcolors=blue,
    pdfsubject={Data Bootcamp @ NYU Stern School of Business},
    pdfauthor={Dave Backus db3@nyu.edu},
    pdfstartview={FitH},
    pdfpagemode={UseNone},
%    pdfnewwindow=true,      % links in new window
%    linkcolor=blue,         % color of internal links
%    citecolor=blue,         % color of links to bibliography
%    filecolor=blue,         % color of file links
%    urlcolor=blue           % color of external links
% see:  http://www.tug.org/applications/hyperref/manual.html
}

%\renewcommand{\thefootnote}{\fnsymbol{footnote}}

% table layout
\usepackage{booktabs}

% section spacing and fonts
\usepackage[small,compact]{titlesec}

% list spacing
\usepackage{enumitem}
\setitemize{leftmargin=*, topsep=0pt}
\setenumerate{leftmargin=*, topsep=0pt}

% attach files to the pdf
\usepackage{attachfile}
    \attachfilesetup{color=0.75 0 0.75}

\usepackage{needspace}
% example:  \needspace{4\baselineskip} makes sure we have four lines available before pagebreak

\usepackage{verbatim}


% Make single quotes print properly in verbatim
\makeatletter
\let \@sverbatim \@verbatim
\def \@verbatim {\@sverbatim \verbatimplus}
{\catcode`'=13 \gdef \verbatimplus{\catcode`'=13 \chardef '=13 }}
\makeatother


% document starts here
\begin{document}
\parskip=\bigskipamount
\parindent=0.0in
\thispagestyle{empty}
{\large Data Bootcamp @ NYU Stern \hfill Dave Backus \& Glenn Okun}


\bigskip\bigskip
\centerline{\Large \bf Topic Outline:  Review \& Applications}
\centerline{Revised: \today}

\section*{Materials}

\begin{itemize}
\item  Today's handouts:  this outline, stickers (as needed)
\item  Posted on {\it Topic outlines \& links\/} page of website (except the stickers)
\end{itemize}

\section*{Preliminaries}

\begin{itemize}
\item Next week's exam
\begin{itemize}
\item Covers:  Python fundamentals 1/2, data input with Pandas, graphics with Matplotlib
\item Like a driving test:  just the essentials
\item Format:  an IPython notebook like the one we use below.  Add answers, email it back to us.
\item Rules:  open book and open internet (wireless permitting), but we recommend a one-page ``cheat sheet''
\end{itemize}

\item {\bf Exercise (review setup)}
\begin{itemize}
\item Put red sticker on your laptop

\item Download IPython notebook

%{\small
\url{https://github.com/DaveBackus/Data_Bootcamp/blob/master/Code/IPython/bootcamp_exam_practice.ipynb} %}

and save Raw file in your \verb|Data_Bootcamp| directory

In short:  GitHub $\Rightarrow$ Code $\Rightarrow$ IPython $\Rightarrow$ \verb|bootcamp_exam_practice.ipynb|
$\Rightarrow$ Raw

\item {\bf We're going to start Jupyter without using Launcher}
\item Go to the command line
\begin{itemize}
\item Windows:  push the Windows key and enter "command prompt".
\item Macs:  click the magnifying glass in the top right and enter "terminal".
\end{itemize}
\item Type:  jupyter notebook [enter]
\item If this starts Jupyter, you're all set.  {\bf If not, let us know.}
\item Replace red sticker with green when you're set
\end{itemize}

%\needspace{3\baselineskip}
\end{itemize}


\section*{Exam practice}

Work your way through the practice exam.  Raise your hand when you get stuck.
\begin{itemize}
\item IPython basics
\item Python fundamentals
\item Data input with Pandas
\item Graphics with Matplotlib
\end{itemize}


\section*{Applications}

{\bf Setup}
\begin{itemize}
\item Put red sticker on your laptop

\item Download IPython notebook

%{\small
\url{https://github.com/DaveBackus/Data_Bootcamp/blob/master/Code/IPython/bootcamp_examples.ipynb} %}

and save Raw file in your \verb|Data_Bootcamp| directory

In short:  GitHub $\Rightarrow$ Code $\Rightarrow$ Lab $\Rightarrow$ \verb|UN_demography.ipynb|
$\Rightarrow$ Raw

\item Start Jupyter -- {\bf from the command line} -- and open notebook
\item Replace red sticker with green when you're set
\end{itemize}

{\bf Demography.} 
We'll spend the rest of the class looking at demographic data.
We do this partly because it's inherently interesting,
partly because it's a good illustration of the research process:
We start with one fact, which suggests followup questions
that drive us to look for other facts,
Repeat as needed.

The code here goes beyond what we've done so far.  
We'll fill that in later, but feel free to ask questions
as we go. 
Topic list:   
\begin{itemize}
\item Aging populations (esp Japan)
\item Fertility (births)
\item Life expectancy
\item Mortality (deaths)
\end{itemize}


\section*{After class}

\begin{itemize}
\item Required
\begin{itemize}
\item Nothing
\end{itemize}
\item Recommended
\begin{itemize}
\item Review book chapters
\item Review code practice
\item Prepare your cheat sheet
\end{itemize}
\end{itemize}

{\vfill
{\bigskip \centerline{\it \copyright \ \number\year \
David Backus, Chase Coleman, and Spencer Lyon @ NYU Stern}%
}}


\end{document}

