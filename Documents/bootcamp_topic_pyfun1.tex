\documentclass[11pt]{article}

\oddsidemargin=0.25truein \evensidemargin=0.25truein
\topmargin=-0.5truein \textwidth=6.0truein \textheight=8.75truein

%\RequirePackage{graphicx}
\usepackage{comment}
\usepackage{hyperref}
\urlstyle{rm}   % change fonts for url's (from Chad Jones)
\hypersetup{
    colorlinks=true,        % kills boxes
    allcolors=blue,
    pdfsubject={Data Bootcamp @ NYU Stern School of Business},
    pdfauthor={Dave Backus db3@nyu.edu},
    pdfstartview={FitH},
    pdfpagemode={UseNone},
%    pdfnewwindow=true,      % links in new window
%    linkcolor=blue,         % color of internal links
%    citecolor=blue,         % color of links to bibliography
%    filecolor=blue,         % color of file links
%    urlcolor=blue           % color of external links
% see:  http://www.tug.org/applications/hyperref/manual.html
}

%\renewcommand{\thefootnote}{\fnsymbol{footnote}}

% table layout
\usepackage{booktabs}

% section spacing and fonts
\usepackage[small,compact]{titlesec}

% list spacing
\usepackage{enumitem}
\setitemize{leftmargin=*, topsep=0pt}
\setenumerate{leftmargin=*, topsep=0pt}

% attach files to the pdf
\usepackage{attachfile}
    \attachfilesetup{color=0.75 0 0.75}

\usepackage{needspace}
% example:  \needspace{4\baselineskip} makes sure we have four lines available before pagebreak

\usepackage{verbatim}


% Make single quotes print properly in verbatim
\makeatletter
\let \@sverbatim \@verbatim
\def \@verbatim {\@sverbatim \verbatimplus}
{\catcode`'=13 \gdef \verbatimplus{\catcode`'=13 \chardef '=13 }}
\makeatother


% document starts here
\begin{document}
\parskip=\bigskipamount
\parindent=0.0in
\thispagestyle{empty}
{\large Data Bootcamp @ NYU Stern \hfill Dave Backus \& Glenn Okun}


\bigskip\bigskip
\centerline{\Large \bf Topic Outline:  Python Fundamentals 1}
\centerline{Revised: \today}


\section*{Materials}

\begin{itemize}
\item  Today's handouts:  this outline, book chapter, code practice, red/green stickers
\item  All posted on {\it Topic list \& links\/} page of website (except the stickers).
\end{itemize}


\section*{Thinking about data}

Not something we'll talk about much in class for a while, but it should be in the back of
your mind over the next 3-4 weeks.  Keep your eyes open for data you want to work with.
Skim the Data sources page of our website.
Ask for help if you find something you can't handle; we're putting together
a collection of programs that read in a variety of datasets, would be happy to add yours
to the list.


\section*{Our approach}

\begin{itemize}
\item One step at a time
\item Ask for help if you need it
\item Newbies:  skip things labelled challenging until you're up to it
\end{itemize}


\section*{Preliminaries}

\begin{itemize}

\item Tools and buzzwords
\begin{itemize}
\item Google fu, Spyder, syntax, calculations, assignments, strings, lists, built-in functions, methods, tab completion, object inspector
\end{itemize}

\item {\bf Exercise}
\begin{itemize}
\item Put red sticker on your laptop
\item Start Spyder
\item Point out the editor and IPython console
\item Open new (empty) file, save as \verb|bootcamp_class_pyfun1.py| in \verb|Data_Bootcamp| directory/folder.  This will serve as your notes for the class.  
%\item Raise your hand if you need help
\item Replace red sticker with green when you're set
\end{itemize}


\item Python programs
\begin{itemize}
\item Syntax:  the rules of Python are less flexible than (say) English
\item Not like Excel: they run line by line, like a book
\item Ours will include:  data input, data management, graphics
\item Examples:  Maddison data, OECD healthcare indicators
\end{itemize}
\end{itemize}


\section*{Python fundamentals 1}

We'll follow the book chapter.
\begin{itemize}
\item Calculations %:  \verb| 2*3, 2 * 3, 2/3, 2^3, 2**3 |
\item Assignments
\item The {\tt print()} function
\item Strings
\item Spyder
\item Help
\item Code cells
\item Comments
\item Quotes
\item Lists
\item Tuples
\item Built-in functions:  {\tt len()}, {\tt type()}, conversions
\item Objects and methods, tab completion
\item Python 2 and 3
\item Review
\begin{itemize}
\item Put red sticker on, replace with green when done
\item Exercises marked challenging are optional
\end{itemize}
\end{itemize}

\section*{After class}

\begin{itemize}
\item Required
\begin{itemize}
%\item Review this chapter
\item Code Practice \#1 due next week (should take about an hour) (we count best 2 of 3)
\end{itemize}
\item Recommended (after every class) 
\begin{itemize}
\item {\bf Write}:  Write down everything you remember without using your notes.  
\item {\bf Review:} Reread the chapter and fill in anything you missed.
\end{itemize}
\end{itemize}

{\vfill
{\bigskip \centerline{\it \copyright \ \number\year \
David Backus, Chase Coleman, and Spencer Lyon @ NYU Stern}%
}}


\end{document}

Second half quiz 

How do we get Python help in Spyder?
What's the diff between 'Spencer' and "Spencer"?
What does this do?  x = 7, x = x -3.  