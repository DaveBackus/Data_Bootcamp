\documentclass[11pt]{exam}

\oddsidemargin=0.25truein \evensidemargin=0.25truein
\topmargin=-0.5truein \textwidth=6.0truein \textheight=8.75truein

%\RequirePackage{graphicx}
\usepackage{comment}
\usepackage{hyperref}
\urlstyle{rm}   % change fonts for url's (from Chad Jones)
\hypersetup{
    colorlinks=true,        % kills boxes
    allcolors=blue,
    pdfsubject={Data Bootcamp @ NYU Stern School of Business},
    pdfauthor={Dave Backus db3@nyu.edu},
    pdfstartview={FitH},
    pdfpagemode={UseNone},
%    pdfnewwindow=true,      % links in new window
%    linkcolor=blue,         % color of internal links
%    citecolor=blue,         % color of links to bibliography
%    filecolor=blue,         % color of file links
%    urlcolor=blue           % color of external links
% see:  http://www.tug.org/applications/hyperref/manual.html
}

%\renewcommand{\thefootnote}{\fnsymbol{footnote}}

% table layout
\usepackage{booktabs}

% section spacing and fonts
\usepackage[small,compact]{titlesec}

% list spacing
\usepackage{enumitem}
\setitemize{leftmargin=*, topsep=0pt}
\setenumerate{leftmargin=*, topsep=0pt}

% attach files to the pdf
\usepackage{attachfile}
    \attachfilesetup{color=0.75 0 0.75}

\usepackage{needspace}
% example:  \needspace{4\baselineskip} makes sure we have four lines available before pagebreak

\usepackage{verbatim}


% Make single quotes print properly in verbatim
\makeatletter
\let \@sverbatim \@verbatim
\def \@verbatim {\@sverbatim \verbatimplus}
{\catcode`'=13 \gdef \verbatimplus{\catcode`'=13 \chardef '=13 }}
\makeatother


% document starts here
\begin{document}
\parskip=\bigskipamount
\parindent=0.0in
\thispagestyle{empty}
{\large Data Bootcamp @ NYU Stern \hfill Dave Backus \& Glenn Okun}


\bigskip\bigskip
\centerline{\Large \bf Data Bootcamp:  Code Practice \#3}
\centerline{Revised: \today}

{\it Answer each of the questions below.
Answers should include any code you wrote to find your answer.  
They can be handwritten, but they must be readable and look professional.}


Run this code to produce the dataframe \texttt{weo}:  
\vspace{-0.15in} 
\begin{verbatim}
import pandas as pd 
data = {'BRA': [13.37, 13.30, 14.34, 15.07, 15.461625, 15.98, 16.10],
        'JPN': [33.43, 31.83, 33.71, 34.29, 35.60, 36.79, 37.39],
        'USA': [48.30, 46.91, 48.31, 49.72, 51.41, 52.94, 54.60], 
        'Year': [2008, 2009, 2010, 2011, 2012, 2013, 2014]}
weo  = pd.DataFrame(data)
\end{verbatim}
\vspace{-0.15in} 
The numbers are GDP per person in thousands of US dollars, 2008 to 2014, 
variable PPPPC in the World Economic Outlook database. 
The code is at the bottom of 
\href{https://github.com/DaveBackus/Data_Bootcamp/blob/master/Code/Python/bootcamp_pandas_1.py}
{this program}; 
you can cut and paste it from there to save time and effort. 


\begin{questions}
\item {\it Review.\/}  Something about:  objects and methods.  Help.  Dictionaries.  Types.  

\item In the code at the top, explain these components:
\begin{parts}
\item The \texttt{import} statement.
\item The data dictionary.   
\item The \texttt{pd.} in the last line. 
\end{parts} 

\item Describe the structure of \texttt{weo}.  What are its dimensions?  
Its column labels? Its row labels?  

\item What ``type'' is \texttt{weo['BRA']}?
What does this mean in non-technical language?   

\item Create a new variable equal to the ratio of Brazilian to US GDP per capita.  
What is its value in 2014?  

\item What method would you use to ``export'' the dataframe as an Excel spreadsheet?  
Verify that it works by opening the spreadsheet you create.  

\item Describe the result of the statement \texttt{t = weo.tail(3)}.  
What kind of object is \texttt{t}?  What does it look like?  

\item How would you create a new dataframe that consists of the first 4 rows of \texttt{weo}?  

\item Set the ``index'' to the \texttt{Year} variable.
Does this change the dimensions of the dataframe?

\item Change the variable names from country codes to country names:  
Brazil, Japan, and the United States.  

\item What method would you use to compute the mean for each country?
What are the means?  

\item {\it Challenging.\/} 
How would you compute the means for each date across countries?  

\item Read the first 10 rows of the 538 college majors data.  
What are the dimensions of the resulting dataframe?  

\item Create a dataframe of the first ten rows of the 538 college majors data 
that contains only the variables numbered \texttt{[2, 15, 16, 17]}.  
(Remind yourself that numbering starts at zero.)
What are the names of these variables?  
{\it Bonus points:\/} What do the last two variables represent?  

\item {\it Challenging.\/} 
The data come sorted by median income, the variable \texttt{Median}.  
How does the ranking change if we sort by \texttt{P75th}?  
What comes in second place?  What do we learn about this major?  

\item Read from your computer...


\end{questions}


\end{document}
