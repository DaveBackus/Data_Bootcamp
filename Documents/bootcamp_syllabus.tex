\documentclass[11pt]{article}

\oddsidemargin=0.25truein \evensidemargin=0.25truein
\topmargin=-0.5truein \textwidth=6.0truein \textheight=8.75truein

%\RequirePackage{graphicx}
\usepackage{comment}
\usepackage{hyperref}
\urlstyle{rm}   % change fonts for url's (from Chad Jones)
\hypersetup{
    colorlinks=true,        % kills boxes
    allcolors=blue,
    pdfsubject={Data Bootcamp @ NYU Stern School of Business},
    pdfauthor={Dave Backus db3@nyu.edu},
    pdfstartview={FitH},
    pdfpagemode={UseNone},
%    pdfnewwindow=true,      % links in new window
%    linkcolor=blue,         % color of internal links
%    citecolor=blue,         % color of links to bibliography
%    filecolor=blue,         % color of file links
%    urlcolor=blue           % color of external links
% see:  http://www.tug.org/applications/hyperref/manual.html
}

%\renewcommand{\thefootnote}{\fnsymbol{footnote}}

% table layout
\usepackage{booktabs}

% section spacing and fonts
\usepackage[small,compact]{titlesec}

% list spacing
\usepackage{enumitem}
\setitemize{leftmargin=*, topsep=0pt}
\setenumerate{leftmargin=*, topsep=0pt}

% attach files to the pdf
\usepackage{attachfile}
    \attachfilesetup{color=0.75 0 0.75}

\usepackage{needspace}
% example:  \needspace{4\baselineskip} makes sure we have four lines available before pagebreak

\usepackage{verbatim}


% Make single quotes print properly in verbatim
\makeatletter
\let \@sverbatim \@verbatim
\def \@verbatim {\@sverbatim \verbatimplus}
{\catcode`'=13 \gdef \verbatimplus{\catcode`'=13 \chardef '=13 }}
\makeatother


% document starts here
\begin{document}
\parskip=\bigskipamount
\parindent=0.0in
\thispagestyle{empty}
{\large Data Bootcamp @ NYU Stern \hfill Dave Backus \& Glenn Okun}


\bigskip\bigskip
\centerline{\Large \bf Data Bootcamp:  Syllabus}
\centerline{Revised: \today}


Rough draft.

\section*{Overview}

Data Bootcamp is about nuts and bolts data analysis. You will learn about economic, financial, and business
data, and enough about computer programming to work with it effectively.
Applications include some or all of: leading economic indicators; emerging market country indicators;
bond and equity returns; stock options; income by zip code; long tail sales data; innovation diffusion curves; and many others.
We will use Python, a popular high-level computer language that's widely used in finance, consulting,
technology, and other parts of the business world.
``High-level'' means it's less painful than most (the hard work is done by the language),
but it's a serious language with extensive capabilities.
``Data analysis'' means primarily graphical descriptions that summarize the data
in ways that are helpful and informative.
``Bootcamp'' is a reminder that expertise takes work. Don't worry, it's worth it.

There are two sections of this course,
one for undergrads (ECON-UB.0232) and one for MBA students (ECON-GB.2313).
The content is similar, but the schedules and teaching teams are different.
See the course website for details.  


\section*{Requirements}

There are no prerequisites.
We welcome students with no prior programming experience and have designed the course with them in mind.
What you will need is the courage and patience to fix computer programs that don't work.
That's a regular occurrence, even for experts;
we will show you how to work your way through it.

Our one requirement is that {\bf you must bring a laptop computer to class\/}.
It should be your own computer, or at least one you can install new programs on.
We will use it constantly in class, writing and correcting (mostly short) programs.


\section*{Course website \& discussion groups}

Eveything you need for the course, including this document, will be posted on
the course website (?? fix link when it's set up):
%
\vspace{-0.2in}
\begin{center}
\url{https://github.com/DaveBackus/Data_Bootcamp#data-bootcamp}.
\end{center}
\vspace{-0.2in}
%
An essential part of this is a google group.


\section*{Deliverables and grades}

The first half of the course is an introduction to computer programming
using the Python programming language.
We focus on those features of Python most useful to data analysis.
The work is front loaded, with three assignments and a quiz
in the first seven weeks.
We think you should do all three assignments --- they're good practice ---
but your grade will be based only on the best two.
The logic behind this plan is to get everyone up to speed quickly
by doing a little work all the time rather than lots of work once in a while.

The quiz will take 75 minutes.
You can bring one page of notes, standard letter paper,
both sides, with anything on it you like.
This will save you from having to memorize things.
It's also a good study tool:  when you decide what to include,
you'll be organizing your thoughts about what you've learned.


The second half of the course is devoted to special topics and a project.
Our goal here is for you to have a piece of work you can show potential employers
to illustrate your skill set.
We build up to a project one step at a time, starting with idea generation and ending
with a professional piece of data collection and analysis.
The structure of the project is laid out in a separate document.

Assignments, whether code practice or parts of the project,
are due at the start of class on the specified dates.
{\bf Due dates are not negotiable.\/}
Anything handed in late will get a grade of zero.

Your final grade will be computed from
\begin{center}
\begin{tabular}{ll}
Assignments (best two of three) & 30\% \\
Quiz        & 20\% \\
Project     & 50\% \\
\end{tabular}
\end{center}
Final grades are not subject to any fixed distribution.
The number of A grades, for example,
will depend only on your performance in the course.
If you make a good-faith effort,
we expect it to be hard to get less than a B.
We will be the sole judges of what constitutes good-faith effort.


\section*{Due dates}

Due dates for assignments are posted on the course website.
{\bf Dates are firm and not open to negotiation.}


\section*{If you have questions}

You can find answers to common questions on the course website.
%Click
%\href{https://github.com/DaveBackus/Data_Bootcamp#data-bootcamp}{here} or go to
%
%\vspace{-0.2in}
%\begin{center}
%\url{https://github.com/DaveBackus/Data_Bootcamp#data-bootcamp}.
%\end{center}
%
%\vspace{-0.2in}
For others, email Dave Backus at
\href{mailto:db3@nyu.edu}{db3@nyu.edu}.



\section*{Policies}

Ethics, disabilities, and many other things are governed by NYU
and Stern policies.
If you have questions about them, please ask.

On graded work:
You may discuss assignments with anyone (in fact, we encourage it),
but anything you submit, including your code, should be your own.
Quizzes should be entirely your own work.

On disabilities:
If you have a qualified disability that requires academic accommodation,
please contact the Moses Center for Students with Disabilities
(\href{http://www.nyu.edu/life/safety-health-wellness/students-with-disabilities.html}{CSD},
212-998-4980) and ask them to
send me a letter verifying your registration and outlining the accommodation they recommend.
If you need to take an exam at the CSD,
you must submit a completed Exam Accommodations Form to them
at least one week prior to the scheduled exam time to be assured accommodation.


{\vfill
{\bigskip \centerline{\it \copyright \ \number\year \
David Backus, Chase Coleman, and Spencer Lyon @ NYU Stern}%
}}


\end{document}

OLD STUFF

\begin{itemize}
\item Data. We live in a world that's ...
\item Programming. To make sense of the data, you need to be able to code:
to write computer programs to put the data in a form that ...
Be forwarned:  you are expected to write programs in this course.
This can be incredibly frustrating and requires enormous patience.
If that's not your thing, you should not take this course.
\item Graphics.
\end{itemize}

