\documentclass[11pt]{exam}

\oddsidemargin=0.25truein \evensidemargin=0.25truein
\topmargin=-0.5truein \textwidth=6.0truein \textheight=8.75truein

%\RequirePackage{graphicx}
\usepackage{comment}
\usepackage{hyperref}
\urlstyle{rm}   % change fonts for url's (from Chad Jones)
\hypersetup{
    colorlinks=true,        % kills boxes
    allcolors=blue,
    pdfsubject={Data Bootcamp @ NYU Stern School of Business},
    pdfauthor={Dave Backus db3@nyu.edu},
    pdfstartview={FitH},
    pdfpagemode={UseNone},
%    pdfnewwindow=true,      % links in new window
%    linkcolor=blue,         % color of internal links
%    citecolor=blue,         % color of links to bibliography
%    filecolor=blue,         % color of file links
%    urlcolor=blue           % color of external links
% see:  http://www.tug.org/applications/hyperref/manual.html
}

%\renewcommand{\thefootnote}{\fnsymbol{footnote}}

% table layout
\usepackage{booktabs}

% section spacing and fonts
\usepackage[small,compact]{titlesec}

% list spacing
\usepackage{enumitem}
\setitemize{leftmargin=*, topsep=0pt}
\setenumerate{leftmargin=*, topsep=0pt}

% attach files to the pdf
\usepackage{attachfile}
    \attachfilesetup{color=0.75 0 0.75}

\usepackage{needspace}
% example:  \needspace{4\baselineskip} makes sure we have four lines available before pagebreak

\usepackage{verbatim}


% Make single quotes print properly in verbatim
\makeatletter
\let \@sverbatim \@verbatim
\def \@verbatim {\@sverbatim \verbatimplus}
{\catcode`'=13 \gdef \verbatimplus{\catcode`'=13 \chardef '=13 }}
\makeatother


% document starts here
\begin{document}
\parskip=\bigskipamount
\parindent=0.0in
\thispagestyle{empty}
{\large Data Bootcamp @ NYU Stern \hfill Dave Backus \& Glenn Okun}


\bigskip\bigskip
\centerline{\Large \bf Data Bootcamp:  Code Practice \#1}
\centerline{Revised: \today}

\begin{questions}
\item Describe what does each of these expressions produces in basic Python if run in order:
\begin{verbatim}
2+5
2 + 5
5**2
x = 3
x = x/2
y = 'abc'
y = 2*y
z = x + y
\end{verbatim}

\item What will be the value of \texttt{y} after running these statements in order? Why?
\begin{verbatim}
x = 3
y = x
x = 10
y
\end{verbatim}

\item What will be the value of \texttt{x} after running these statements in order? Why?
\begin{verbatim}
x = 3
x = x + 3
x -= 3
x
\end{verbatim}

\item Will this code run without errors? Why?
\begin{verbatim}
x = 3
y = '24'
y / x
\end{verbatim}

What about this?
\begin{verbatim}
x = 3
y = 24
y / x
\end{verbatim}

\item Will this code run without errors? Why? If it does run, what is the value of \texttt{x}?
\begin{verbatim}
x = "I am a #string"  # Whoa, a string
\end{verbatim}

\item Will this code run without errors? Why? If it does run, what is the value of \texttt{z}?
\begin{verbatim}
x = [1, 2, 3]
y = [42, 43]
z = x + y
\end{verbatim}

\item What does each line do below? Why?
\begin{verbatim}
type(42)
type(42.0)
type('42.0')
type("42.0")
type("""42.0""")
type([1, 2])
type([1] + [2])
type(1 + 2)
type(print)

# warning, this one is tricky
type(float(str(int(str(1)))))
\end{verbatim}

\item What does each line do below? Why?
\begin{verbatim}
len([113423])
len("113423")
len(113423)  # watch out!
\end{verbatim}

\item How would you convert \texttt{x} to title case (first letter of each word capitalized)? Hint: Use tab completion to find a method
\begin{verbatim}
x = "luke, i am your father"
\end{verbatim}

\item (challenging) Have python figure out how many words are in this string. Hint, use a method and a function. % Answer len(x.split())
\begin{verbatim}
x = "Have python figure out how many words are in this string."
\end{verbatim}

\item What python code would tell you how to write \texttt{x} below as a fraction of two integers? What are those integers? Hint: use a method % Answer: x.as_integer_ratio()
\begin{verbatim}
x = 0.1234
\end{verbatim}

\item What does the \texttt{join} method of a string do? Hint: Use spyder's IPython console and a question mark

\item (challenging) Convert \texttt{x} into a string. Hint: use \texttt{join} % Answer: "".join(x)
\begin{verbatim}
x = ['H', 'e', 'l', 'l', 'o', ',', ' ', 'W', 'o', 'r', 'l', 'd', '!']
\end{verbatim}



\end{questions}

{\vfill
{\bigskip \centerline{\it \copyright \ \number\year \
David Backus, Chase Coleman, and Spencer Lyon @ NYU Stern}%
}}


\end{document}
