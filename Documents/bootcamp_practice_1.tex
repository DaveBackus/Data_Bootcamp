\documentclass[11pt]{exam}

\oddsidemargin=0.25truein \evensidemargin=0.25truein
\topmargin=-0.5truein \textwidth=6.0truein \textheight=8.75truein

%\RequirePackage{graphicx}
\usepackage{comment}
\usepackage{hyperref}
\urlstyle{rm}   % change fonts for url's (from Chad Jones)
\hypersetup{
    colorlinks=true,        % kills boxes
    allcolors=blue,
    pdfsubject={Data Bootcamp @ NYU Stern School of Business},
    pdfauthor={Dave Backus db3@nyu.edu},
    pdfstartview={FitH},
    pdfpagemode={UseNone},
%    pdfnewwindow=true,      % links in new window
%    linkcolor=blue,         % color of internal links
%    citecolor=blue,         % color of links to bibliography
%    filecolor=blue,         % color of file links
%    urlcolor=blue           % color of external links
% see:  http://www.tug.org/applications/hyperref/manual.html
}

%\renewcommand{\thefootnote}{\fnsymbol{footnote}}

% table layout
\usepackage{booktabs}

% section spacing and fonts
\usepackage[small,compact]{titlesec}

% list spacing
\usepackage{enumitem}
\setitemize{leftmargin=*, topsep=0pt}
\setenumerate{leftmargin=*, topsep=0pt}

% attach files to the pdf
\usepackage{attachfile}
    \attachfilesetup{color=0.75 0 0.75}

\usepackage{needspace}
% example:  \needspace{4\baselineskip} makes sure we have four lines available before pagebreak

\usepackage{verbatim}


% Make single quotes print properly in verbatim
\makeatletter
\let \@sverbatim \@verbatim
\def \@verbatim {\@sverbatim \verbatimplus}
{\catcode`'=13 \gdef \verbatimplus{\catcode`'=13 \chardef '=13 }}
\makeatother


%\printanswers

% document starts here
\begin{document}
\parskip=\bigskipamount
\parindent=0.0in
\thispagestyle{empty}
{\large Data Bootcamp @ NYU Stern \hfill Dave Backus \& Glenn Okun}


\bigskip\bigskip
\centerline{\Large \bf Data Bootcamp:  Code Practice \#1}
\centerline{Revised: \today}

{\it Answer each of the questions below.
Answers should include any code you write to find the answer.
They can be handwritten, but they must be readable and look professional.}

\begin{questions}
\item Describe and explain what each of these expressions produces in basic Python:
\begin{verbatim}
2+5
2 + 5
2*5
2/5
2**5
\end{verbatim}

\begin{solution}
In order:  7, 7 (spaces don't matter), 10, 0.4, 32 (5 is an exponent).
\end{solution}

\item What is the value of \texttt{x} after running these statements in order? Why?
\begin{verbatim}
x = 7
x = x + 3
\end{verbatim}

\begin{solution}
First, we set {\tt x} equal to 7, then we set it equal to 7 + 3 or 10.
The second line looks funny as mathematics, but the rhs takes the current value of {\tt x}, namely 7,
adds 3 to it, and assigns the result to {\tt x}.
\end{solution}

\item What is the value of \texttt{y} after running these statements in order?
Of \texttt{x}?  Why?
\begin{verbatim}
x = 3
y = x
x = 10
\end{verbatim}

\item Does this code run without error?  If so, what does it produce?  If not, explain why.
\begin{verbatim}
x = 3
x = x/2
y = 'abc'
z = y + y
print(x, z)
\end{verbatim}

\item Does this code run without error?  If so, what does it produce?  If not, explain why.
\begin{verbatim}
x = 3
x = x/2
y = 'abc'
z = x + y
print(x, z)
\end{verbatim}

\item Does this code run without error?  If so, what does it produce?  If not, explain why.
\begin{verbatim}
x = 3
y = 24
z = y / x
print(x, y, z, sep=' | ')
\end{verbatim}

\item Does this code run without error?  If so, what does it produce?  If not, explain why.
\begin{verbatim}
x = 3
y = '24'
z = y / x
print(x, z)
\end{verbatim}

\item Does this code run without error?  If so, what does it produce?  If not, explain why.
\begin{verbatim}
x = "I am a #string"  # Whoa, a string!
\end{verbatim}

\item Does this code run without error?  If so, what does it produce?  If not, explain why.
\begin{verbatim}
x = [1, 2, 3]
y = [42, 43]
z = x + y
print(z)
\end{verbatim}

\item Does this code run without error?  If so, what does it produce?  If not, explain why.
\begin{verbatim}
x = [1, 2, 3]
y = 42
z = x + y
\end{verbatim}

\item What ``types'' are
\begin{verbatim}
x1 = 12
x2 = 12.0
x3 = '12.0'
x4 = [12]
x5 = [12, 12.0, '12.0']
\end{verbatim}

\item Explain the result of each line:
\begin{verbatim}
type(42)
type(42.0)
type('42.0')
type("42.0")
type("""42.0""")
type([1, 2])
type([1] + [2])
type(1 + 2)
type(print)
\end{verbatim}

\item Describe and explain the result of this statement:
\begin{verbatim}
type(float(str(int('1234'))))
\end{verbatim}

\needspace{2\baselineskip}
\item Describe and explain the result of this statement:
\begin{verbatim}
type(int(float('12.34')))
\end{verbatim}

\item Explain each line:
\begin{verbatim}
len([1234])
len("1234")
len(1234)
\end{verbatim}

\item What are the type and length of \texttt{x = []}?


% http://tex.stackexchange.com/questions/48632/underscores-in-words-text
\item Convert the string \verb|x = 'abcde'| to a list.  What does it look like?

\item {\it Challenging.\/}
Consider the integer \texttt{x = 1234}.
\begin{parts}
\item Convert \texttt{x} to a floating point number.
\item Convert \texttt{x} to a string.
\item Convert \texttt{x} to the list
\begin{verbatim}
['1', '2', '3', '4']
\end{verbatim}

\end{parts}

\item {\it Challenging.\/}
How would you convert \texttt{x} to ``title case'' (first letter of each word capitalized)?
{\it Hint:\/} Use tab completion to find an appropriate method.
\begin{verbatim}
x = "luke, i am your father"
\end{verbatim}

\item {\it Challenging.\/} Consider the string
\begin{verbatim}
x = "How many characters and words are in this string?"
\end{verbatim}
\begin{parts}
\item How many characters does \texttt{x} contain?
\item Convert \texttt{x} to a list of individual characters.
\item Convert \texttt{x} to a list of individual words.
{\it Hint:\/}  Use tab completion to find a method that splits \texttt{x}
into pieces.
\item How many words does \texttt{x} contain?
\end{parts}
% Answer len(x.split())


%\item Explain the following
%\begin{parts}
%\item Integer
%\item String
%\item Floating point number
%\item Type conversion.
%\end{parts}


%\item What python code would tell you how to write \texttt{x} below as a fraction of two integers? What are those integers? Hint: use a method % Answer: x.as_integer_ratio()
%\begin{verbatim}
%x = 0.1234
%\end{verbatim}

%\item What does the \texttt{join} method of a string do?
%{\it Hint:\/} Use Spyder's IPython console and a question mark.
%
%\item (challenging) Convert \texttt{x} into a string. Hint: use \texttt{join} % Answer: "".join(x)
%\begin{verbatim}
%x = ['H', 'e', 'l', 'l', 'o', ',', ' ', 'W', 'o', 'r', 'l', 'd', '!']
%\end{verbatim}

\end{questions}

{\vfill
{\bigskip \centerline{\it \copyright \ \number\year \
David Backus, Chase Coleman, and Spencer Lyon @ NYU Stern}%
}}


\end{document}
