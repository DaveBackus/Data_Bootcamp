\documentclass[handout]{beamer}
%\documentclass[slidestop,xcolor=pst,dvips]{beamer}
\usepackage{pstricks}
\usepackage{pgf,pgfarrows,pgfnodes}
%\usepackage{pgfpages}
\usepackage{graphicx}
\usepackage{epstopdf}
\usepackage{subfigure}
\usepackage{pgflibraryshapes}
\usepackage{pgflibrarysnakes}
\usepackage{pslatex}
\usepackage{amssymb}
\usepackage{amsfonts}
\usepackage{amsmath}
\usepackage{textcomp}
\usepackage{comment}
\usepackage{booktabs}
%\usepackage{tikz}
%\usepackage{beamerthemeBerkeley}
% Use either the one above or the one below
%\usetheme{Montpellier}
%\usetheme{Berkeley}
%\usetheme{Goettingen}
%\usetheme{Hannover}
\usetheme{default}

\newcommand{\eps}{\varepsilon}

%\setbeamerfont{frametitle}{size=\LARGE}
%\setbeamercolor{frametitle}{fg!100!bg}
\setbeamercolor{mycolor}{parent=palette primary}
{\usebeamercolor{mycolor}}
\setbeamercolor{shad}{fg=black,bg=black!0}
\setbeamercolor{frametitle}{bg=gray!5}

\setbeamertemplate{frametitle}{

  \vskip6pt
  %\begin{beamercolorbox}[rounded=true,shadow=true, center]{}
%  \LARGE\insertframetitle
%  \end{beamercolorbox}
   %\shadowbox{\usebeamercolor{shad} \LARGE\insertframetitle}
%  \LARGE\insertframetitle
%  \end{beamercolorbox}
    %\usebeamercolor{shad}
    \begin{beamerboxesrounded}[upper=block body,lower=block body,shadow=true]{}
    \begin{center}
    \LARGE\insertframetitle
    \end{center}
     \end{beamerboxesrounded}

  %\vskip-2pt
  %\insertvrule{1.0pt}{mycolor.fg!90!bg}%
}

\definecolor{DB}{rgb}{0,0.0,0.8}
\definecolor{DB1}{rgb}{0.5, 0, 0.2}

%\logo{\includegraphics[height=0.5cm]{logo_skyline_large.eps}}
%\logo{\includegraphics[height=0.5cm]{lse.eps}}
\setbeamersize{description width=5pt}
\setbeamertemplate{items}[ball]
\setbeamertemplate{blocks}[rounded][shadow=true]
\setbeamertemplate{navigation symbols}{}
\setbeamertemplate{footline}[frame number]

\def\stackunder#1#2{\mathrel{\mathop{#2}\limits_{#1}}}

%\title{Disasters Implied by Equity Index Options}
%\author{David Backus\inst{1} \and Mikhail Chernov\inst{2} \and Ian Martin\inst{3} }
%\institute{\inst{1}NYU Stern and NBER \and \inst{2}London Business School and CEPR \and \inst{3}Stanford GSB and NBER}
%\vspace{7cm}
%\date{CEPR/Gerzensee Asset Pricing \\ July 2009}

%****************************************************************************
\begin{document}

%\frame{\titlepage}

%\begin{comment}
%\begin{frame}
\vspace*{0.5in}
\centerline{\textcolor{DB}{\Large\bf Identifiying Taylor Rules}}
\centerline{\textcolor{DB}{\Large\bf in macro-finance models}}
%\centerline{\textcolor{DB}{\Large\bf Representative Agent Models }}

\bigskip\bigskip%\bigskip%\bigskip
\centerline{David Backus (NYU), Mikhail Chernov (UCLA), }
\centerline{and Stanley Zin (NYU)}

\bigskip\bigskip%\bigskip%\bigskip
\centerline{Tom Cooley Conference @ NYU $|$ October 2013}

\vfill
{\tiny This version: \today}


%\end{document}
%\begin{comment}




% --------------------------------------------------------------------------
\begin{frame}
\frametitle{The Taylor rule}

\begin{itemize} \itemsep=\bigskipamount
   \item Taylor rule (TR) is a central equation in many models
      \begin{eqnarray*}
            i_t = r + \tau \pi_t + s_t,
      \end{eqnarray*}
where $i_t$ is the policy rate (nominal), $r$ is the long-term real rate, $\pi_t$ is inflation, and $s_t$ is a monetary policy (MP) shock.
\pause
  \item In macro models it affects inflation, output
  \item In finance models it affects bond yields and prices of other assets
  \item Can we identify $\tau$ ?
\end{itemize}
\end{frame}

% --------------------------------------------------------------------------
\begin{frame}
\frametitle{Identification}

\begin{itemize} \itemsep=\bigskipamount
  % \item The statistical question of identification is how many and which parameters of a given model can be estimated given a specific (possibly infinite) sample
  % \item The question that we are interested in is whether, given the required statistical identification assumptions, one can estimate $\tau$ or infer it from the parameters that can be estimated
   \item The consensus: the answer is ``No''
       \begin{itemize}
            \item Finance: affine term structure models simply do not have enough structure to say anything useful about the economics of the problem
            \item Macro: the elaborate structure does not help in identifying $\tau$
       \end{itemize}
   \pause
    \item What we do:
        \begin{itemize}
            \item Revisit the question systematically, in the context of forward-looking models
            \item  Outline when $\tau$ can and cannot be identified
          \end{itemize}
\end{itemize}


\end{frame}

% --------------------------------------------------------------------------
\begin{frame}
\frametitle{Outline}

\begin{itemize} \itemsep=\bigskipamount
  \item Two examples where TR is not identified
  \item Rational expectations solutions and identification
  \item The ``typical'' setup: the Phillips curve
  \item Unobserved state
\end{itemize}



\end{frame}

% --------------------------------------------------------------------------
\begin{frame}
\frametitle{The setup}

\begin{itemize} \itemsep=\bigskipamount
\item States
\begin{align*}
   x_{t+1} = A x_t + C w_{t+1}
\end{align*}
\item Variance $V_x$ solves:
 \begin{align*}
   V_x = A V_x A^{\top} + CC^{\top}
\end{align*}
\item Initially assume that $x_t$ is observable, relax later
\item Shocks
\begin{align*}
    s_{it}  = d_i^\top x_t
\end{align*}
\item Shocks are not observable when $d_i$'s are unknown
\end{itemize}

\end{frame}

% --------------------------------------------------------------------------
\begin{frame}
\frametitle{Cochrane's example}

\begin{itemize} \itemsep=\bigskipamount
\item Equations %(``Taylor rule'')
\begin{align*}
    i_t &\;\;=\;\; E_t \pi_{t+1}  \tag{Fisher equation} \\
    i_t &\;\;=\;\;  \tau \pi_{t} + s_t \tag{Taylor rule} \\
\end{align*}
\item The forward-looking equation
\begin{align*}
   E_t \pi_{t+1} = \tau \pi_{t} + s_t
\end{align*}
\item Guess $\pi_t=b^\top x_t$
\item Solution: $b^\top=-d^{\top}(\tau I-A)^{-1}$
\item OLS of $i_t$ on $\pi_t$ produces $(b^{\top}V_x b)^{-1} b^{\top} A V_x b$ instead of $\tau$
%\item Because $x_t$ is observed:
%     \begin{itemize}
%        \item Can estimate $A$ from VAR on $x_t$
%        \item Can estimate $b$ from OLS of $\pi_t$ on $x_t$
%     \end{itemize}
 \item Moreover,  cannot establish $\tau$ without knowing $d$
\end{itemize}

\end{frame}

% --------------------------------------------------------------------------
\begin{frame}
\frametitle{Affine term structure example}

\begin{itemize} \itemsep=\bigskipamount
\item Sims and Zha (2006): \\

\medskip
\begin{itemize}
\item[]
\parbox[c]{0.85\textwidth}
{\it \raggedright
The ... problem ... is that the Fisher relation is always lurking in the background. The Fisher
relation connects current nominal rates to expected future inflation rates and to real interest
rates[.] ... So one might easily find an equation that had the form of the forward-looking
Taylor rule, satisfied the identifying restrictions, but was something other than a policy
reaction function.%\\
%``The question is,'' said Alice, ``whether you can make words mean
%so many different things.'' \\
%``The question is,'' said Humpty Dumpty, ``which is to be master ---
%that's all.''
}
\end{itemize}
\pause
\item Equations %(``Taylor rule'')
\begin{align*}
     m^{\$}_{t+1} \;\;=\;\;  - \lambda^\top \lambda/2 - \delta^\top x_{t} + \lambda^\top w_{t+1}  \tag{Pricing kernel} \\
     i_t  \;\;=\;\; - \log E_t \exp(m^{\$}_{t+1})
        \;\;=\;\; \delta^\top x_t \tag{Euler equation} \\
\end{align*}
\item  If $x_{it}=\pi_t,$ one is tempted to conclude that $\delta_i=\tau.$ Is it?
%   \begin{itemize}
%        \item As in the first example, we can estimate $b$ from OLS of $\pi_t$ on $x_t$
%       \item The Taylor rule: $i_t=\tau \pi_t + s_t = \tau b^{\top} x_t + d^{\top} x_t$
%  \end{itemize}
  \item OLS of $i_t$ on $x_t$ produces $\delta^{\top}=\tau b^{\top} + d^{\top},$ so $\delta_i \ne \tau_i$
  \item Moreover,  cannot establish $\tau$ without knowing $d$
\end{itemize}

\end{frame}

% --------------------------------------------------------------------------
\begin{frame}
\frametitle{Follow-up questions}

\begin{itemize} \itemsep=\bigskipamount
\item Would extra shocks help?
   \begin{align*}
    i_t &\;\;=\;\; E_t \pi_{t+1} +s_{1t} \\
    i_t &\;\;=\;\;  \tau \pi_{t} + s_{2t}
\end{align*}
\begin{itemize}
      \item If $s_{it}$'s are independent, then $s_{1t}$ can be used as an instrument for $\pi_t$
      \item Note the independence assumption -- more generally, a restriction
\end{itemize}
\item Would financial data (e.g., the yield curve) help?
\begin{itemize}
      \item We can estimate the state, but the state is assumed to be observed in the examples
\end{itemize}
\end{itemize}

\end{frame}

% --------------------------------------------------------------------------
\begin{frame}
\frametitle{A representative-agent model}
\begin{footnotesize}
\begin{itemize} %\itemsep=\bigskipamount
\item Equations %(``Taylor rule'')
\begin{align*}
    i_t &\;\;=\;\; - \log E_t \exp(m_{t+1}-\pi_{t+1}) \tag{Euler/Fisher equation} \\
        m_t &\;\;=\;\;  - \rho - \alpha g_t  \tag{Pricing kernel}\\
            g_t &\;\;=\;\;  g + s_{1t}  \tag{Consumption growth / output gap} \\
    i_t &\;\;=\;\;  \tau \pi_{t} + s_{2t} \tag{Taylor rule}
\end{align*}
\pause
\item Guess $\pi_t=b^\top x_t$
\item First three equations imply:
\begin{eqnarray*}
   i_t &=&  \rho+\alpha g - V_m/2  + a^\top x_t \\
    a^\top &=& (\alpha d_1^\top + b^\top)  A  \\
    V_m     &=& a^\top C C^\top a
\end{eqnarray*}
\item Solution: $b^\top \;\;=\;\; (\alpha d_1^\top - d_2^\top) (\tau I -  A )^{-1} \Leftrightarrow     a^\top = \tau b^\top + d_2^\top$
\item Estimate $a$ and $b$ from OLS of $i_t$ and $\pi_t$ on $x_t$
 \item We have $n$ knowns $a$ and $n+1$ unknowns  $(\tau,d_2)$
 \item One exclusion restriction is sufficient to identify $\tau,$ e.g., independence of shocks $s_{1t}$ and $s_{2t}$, $d_1^{\top}V_x d_2=0.$
\end{itemize}
\end{footnotesize}
\end{frame}

% --------------------------------------------------------------------------
\begin{frame}
\frametitle{An exponential-affine model}
\begin{footnotesize}
\begin{itemize} %\itemsep=\bigskipamount
\item Equations %(``Taylor rule'')
\begin{align*}
    i_t &\;\;=\;\; - \log E_t \exp(m_{t+1}-\pi_{t+1}) \tag{Euler/Fisher equation} \\
        m_t &\;\;=\;\;  -\rho - s_{1t} + \lambda^\top w_{t+1}   \tag{Pricing kernel}\\
%            g_t &\;\;=\;\;  g + s_{1t}  \tag{Consumption growth / output gap} \\
    i_t &\;\;=\;\;  \tau \pi_{t} + s_{2t} \tag{Taylor rule}
\end{align*}
\pause
\item Guess $\pi_t=b^\top x_t$
\item First two equations imply:
\begin{eqnarray*}
   i_t &=&  \rho - V_m/2  + a^\top x_t \\
    a^\top &=& \delta^\top A + b^\top  \\
    V_m     &=& (\lambda^\top - b^\top C)(\lambda - C^\top b)
\end{eqnarray*}
\item Solution: $b^\top \;\;=\;\;  (\delta^\top - d^\top) (\tau I -  A )^{-1}  \Leftrightarrow     a^\top = \tau b^\top + d^\top$
\item Estimate $a$ and $b$ from OLS of $i_t$ and $\pi_t$ on $x_t$
 \item We have $n$ knowns $a$ and $n+1$ unknowns  $(\tau,d)$
 \item One exclusion restriction is sufficient to identify $\tau,$ e.g., $d_{2i}=0.$
\end{itemize}
\end{footnotesize}
\end{frame}

% --------------------------------------------------------------------------
\begin{frame}
\frametitle{Observations}

\begin{itemize} \itemsep=\bigskipamount
    \item OLS of $i_t$ on $\pi_t$ does not recover $\tau$
    \item The ``right'' OLS helps to recover $\tau:$ separate OLS of $i_t$ and $\pi_t$ on state
     \item  One restriction on MP shock suffices to identify the Taylor rule
     \item Note: it does not matter whether they are shocks in the Euler equation or not
     \item This success reflects what Hansen and Sargent call the hallmark of rational expectations models: that cross-equation restrictions connect the parameters in one equation to those in the others
     \item What about more elaborate models?
\end{itemize}

\end{frame}

% --------------------------------------------------------------------------
\begin{frame}
\frametitle{A Phillips curve}
\begin{footnotesize}
\begin{itemize} %\itemsep=\bigskipamount
\item Equations %(``Taylor rule'')
\begin{align*}
    i_t &\;\;=\;\; - \log E_t \exp(m_{t+1}-\pi_{t+1}) \tag{Euler equation} \\
        m_t &\;\;=\;\;  - \rho - \alpha g_t  \tag{Pricing kernel}\\
            \pi_t &\;\;=\;\;  \beta E_t \pi_{t+1} + \kappa g_t + s_{1t}  \tag{Phillips curve} \\
    i_t &\;\;=\;\;  \tau_1 \pi_{t} + \tau_2 g_t+ s_{2t} \tag{Taylor rule}
\end{align*}
\pause
\item Guess $\pi_t=b^\top x_t$ and $g_t = c^{\top} x_t$
\item First two equations imply:
\begin{eqnarray*}
   i_t &=&  \rho - V_m/2  + a^\top x_t \\
    a^\top &=& (\alpha c^\top + b^\top)  A  \\
    V_m     &=& a^\top C C^\top a
\end{eqnarray*}
\item Last two equations imply:
\begin{eqnarray*}
           b^\top &=& \beta  A  b^\top + \kappa c^\top + d_1^\top \\
                     a^\top &=& \tau_1 b^\top + \tau_2 c^\top + d_2^\top
\end{eqnarray*}
%\item Solution: $b^\top \;\;=\;\; (\alpha d_1^\top - d_2^\top) (\tau I -  A )^{-1} \Leftrightarrow     a^\top = \tau b^\top + d_2^\top$
\item Estimate $a$, $b$ and $c$ from OLS of $i_t$, $\pi_t$ and $g_t$ on $x_t$
 \item In the 2nd eq-n we have $n$ knowns $a$ and $n+2$ unknowns  $(\tau_1, \tau_2,  d_2)$
 \item Two exclusion restrictions is sufficient to identify $\tau_1$ (and $\tau_2$), e.g., independence of shocks $s_{1t}$ and $s_{2t}$, $d_1^{\top}V_x d_2=0.$
\end{itemize}
\end{footnotesize}
\end{frame}

% --------------------------------------------------------------------------
\begin{frame}
\frametitle{Unobserved state}

\begin{itemize} \itemsep=\bigskipamount
    \item Until now we have assumed observed state
        \begin{itemize}
            \item This assumption allowed us to run the OLS critical for identification
        \end{itemize}
      \item What changes when the state has to be estimated?
        \begin{itemize}
            \item Nothing
        \end{itemize}
        \item Have to discuss identification and estimation of the state
\end{itemize}

\end{frame}

% --------------------------------------------------------------------------
\begin{frame}
\frametitle{Identifying the state}

\begin{itemize} %\itemsep=\bigskipamount
    \item The state can be ``rotated'', $\hat{x}_t = T x_t,$ without affecting implications for observed variables
    \item The new state is:
    \begin{eqnarray*}
    \hat{x}_{t+1} &=& T A T^{-1} \hat{x}_t + T C w_{t+1}
            \;\;=\;\; \hat{A} \hat{x}_t + \hat{C} w_{t+1} .
\end{eqnarray*}
   \item The interest rate is $ i_t =r+a^\top T^{-1} \hat{x}_t = r + \hat{a}^{\top} \hat{x}_t$
   \item Similarly, $ \hat{b}^{\top} = b^\top T^{-1}$
   \item The Taylor rule  implies
\begin{eqnarray*}
    \hat{a}^{\top} &=& \tau \hat{b}^{\top} + \hat{d}^{\top}_2 ,
\end{eqnarray*}
  \item Identification of $\tau$ is exactly the same as before
\item Have to impose restrictions to fix a state: $C=I$ and lower-triangular $A$ is a common choice
\end{itemize}

\end{frame}

% --------------------------------------------------------------------------
\begin{frame}
\frametitle{Transformation-invariant restrictions}

\begin{itemize} \itemsep=1.25 \bigskipamount
    \item The impact on restrictions: $d_2^{\top} e=0$ translates into $\hat{d}^{\top}_2 \hat{e} =0$ with  $\hat{e}=Te.$
  \item If $T$ is unknown, can we deduce $\hat{e}$?
    \item Ex. 1: The TR shocks is uncorrelated with the other (Euler equation) shock -- standard in the New-Keynesian literature
    \begin{eqnarray*}
         \hat{d}_2^{\top} E(\hat{x}\hat{x}^{\top})\hat{d_1}=(d_2^{\top}T^{-1})(TV_xT^{\top})(d_1^{\top}T^{-1})^{\top}=d_2^{\top}V_x d_1
    \end{eqnarray*}
    \item Ex. 2: In the optimal monetary policy setting all variables are affected by $s_{1t},$ thus $s_{2t}=ks_{1t}$
        \begin{itemize}
            \item The implied restriction is $d_2^{\top}-kd_1^{\top}=0$
            \item In terms of the transformed state:
            \begin{eqnarray*}
                 \hat{d}_2^{\top}-k\hat{d}_1^{\top}=d_2^{\top}T^{-1}-kd_1^{\top}T^{-1} = (d_2^{\top}-kd_1^{\top})T^{-1}=0
            \end{eqnarray*}
        \end{itemize}
\end{itemize}

\end{frame}


% --------------------------------------------------------------------------
\begin{frame}
\frametitle{Estimating the state}

\begin{itemize} %\itemsep=\bigskipamount
    \item Introduce a measurement equation:
    \begin{eqnarray*}
    y_t &=& G x_t + H v_t .
        %\label{eq:meas}
\end{eqnarray*}
    \item The Kalman filter, ${x}_{t|t}=E(x_t|y^t),$ recovers all states when $(A,G)$ is observable:
    \begin{eqnarray*}
    rank
    \left[
    \begin{array}{c}
        G \\ G  A  \\ \vdots \\ G  A ^{n-1}
    \end{array}
    \right] =n
\end{eqnarray*}
\item As a result, $i_t=a^{\top}{x}_{t|t}+a^{\top}\varepsilon_t,$ $\varepsilon_t\perp y^t$
\item All the earlier logic applies still by replacing $x_t$ with ${x}_{t|t}$
\end{itemize}

\end{frame}

% --------------------------------------------------------------------------
\begin{frame}
\frametitle{Example: Term Structure}

\begin{itemize} %\itemsep=\bigskipamount
    \item $q_t^h$ is the price at date $t$ if a claim to one dollar at $t+h$
    \item Forward rates are defined by $ f^h_t = \log (q^h_t/q^{h+1}_t),$ $f^0_t=i_t$
   \item In our model, $f^h_t = a^{\top} A^h x_t$
   \item The vector $f_t$ of the first $h$ forward rates has the form
\begin{eqnarray*}
    f_t  \;\;=\;\; \left[  \begin{array}{c} f^0_t \\ f^1_t \\ \vdots \\ f^{h-1}_t \end{array} \right]
        &=&
    \left[  \begin{array}{c} a^\top \\ a^\top A \\ \vdots \\ a^\top A^{h-1} \end{array} \right]         \left[ x_t \right]
    \;\;=\;\; U x_t .
\end{eqnarray*}
 \item If $U$ is invertible, we can use the vector of forward rates as the state. %The same holds for yields, defined by $ i^h_t = h^{-1} \sum_{j=1}^h f^{j-1}_t $.
 \item Usually $x_t$ is low-dimensional and $dim(f_t)$ is large. In these cases, researchers attach measurement errors to all forward rates and  use  the Kalman filter to estimate the state.
\end{itemize}

\end{frame}

% --------------------------------------------------------------------------
\begin{frame}
\frametitle{Example: Surveys}

\begin{itemize} %\itemsep=\bigskipamount
    \item Rich data on survey forecasts of macro and financial variables:
    \begin{eqnarray*}
     F_t(i_{t+h})&=&E_t(i_{t+h})+v_t=a^{\top} A^h x_t + v_t \\
   F_t(\pi_{t+h})&=&E_t(\pi_{t+h})+v_t=b^{\top} A^h x_t + v_t \\
    F_t(g_{t+h})&=&E_t(g_{t+h})+v_t=c^{\top} A^h x_t + v_t
\end{eqnarray*}
\item Can use Kalman filter to estimate the state
\end{itemize}

\end{frame}




% --------------------------------------------------------------------------
\begin{frame}
\frametitle{Conclusion}

\begin{itemize} \itemsep=\bigskipamount
    \item Brute-force identification of the Taylor rule is impossible
    \item Rational expectations framework brings information from the whole system to bear on the TR coefficients
    \item We offer a constructive approach towards identification
    \item In general, one needs exclusion restrictions on  MP shocks to identify TR; typical models impose more than what's required
\end{itemize}

\end{frame}

\end{document}

