\documentclass{beamer}
%\documentclass[leqno]{beamer}

\mode<presentation>

\usepackage{graphicx}
\usepackage{comment}
\usepackage{booktabs}
\usepackage{amsmath}

\hypersetup{
%    colorlinks=true,        % kills boxes
%    allcolors=blue,
    pdftitle={Identification},
    pdfauthor={Backus, Chernov, \& Zin},
    pdfcreator={Dave Backus @ NYU},
    pdfstartview={FitH},
    pdfpagemode={UseNone},
%    pdfnewwindow=true,      % links in new window
%    linkcolor=blue,         % color of internal links
%    citecolor=blue,         % color of links to bibliography
%    filecolor=blue,         % color of file links
%    urlcolor=blue           % color of external links
% see:  http://www.tug.org/applications/hyperref/manual.html
}
%\usetheme{Frankfurt}
%\usetheme{Warsaw}
\usetheme[secheader]{Boadilla}

%\setbeamercovered{transparent}
%\setbeameroption{show notes}
\setbeameroption{hide notes}

\special{landscape}

% Basic input for footer
\title[Identification]{}
\author[Backus, Chernov, \& Zin]{}
\institute[NYU, LSE]{}
\date[]{\today}


%****************************************************************************
\begin{document}

\vspace*{0.75in}
\centerline{\Large\bf Identifying Macroeconomic Parameters}
\vspace*{0.05in}
\centerline{\Large\bf With Forecasts, Asset Prices, and Structure}


\bigskip\bigskip\bigskip%\medskip
\centerline{David Backus, Mikhail Chernov,}
\centerline{and Stanley Zin}

\bigskip\bigskip\medskip%\bigskip
\centerline{NYU Macro Lunch $|$ March 8, 2012}

\vfill
{\tiny \hspace{7pt}This version: \today}
{\hfill \includegraphics[scale=0.25]{stern_logo3.eps}}

%\end{document}
%\begin{comment}
% --------------------------------------------------------------------------
\begin{frame}
\frametitle{The problem}
\begin{itemize} \itemsep=\bigskipamount
\item Forward-looking difference equation %(``Taylor rule'')
\begin{eqnarray*}
    y_t &=& \lambda E_t y_{t+1} + x_t \phantom{xxx}
\end{eqnarray*}
\item Questions \\
\begin{itemize}
\item How do we identify $\lambda$?
\item Do we need to observe $x$?
\item Are forecasts and asset prices helpful? Cross-equation restrictions?
\end{itemize}
\item Answers \\
\begin{itemize}
\item We need a model and ``enough'' observables and structure 
\item We need to observe something:  there's no free lunch here
\item Forecasts, asset prices, and economic structure all help
\end{itemize}
\end{itemize}
\end{frame}
%\end{document}

% --------------------------------------------------------------------------
\begin{frame}
\frametitle{Rules of the game}
\begin{itemize} \itemsep=\bigskipamount
\item Economic model \\
\begin{itemize}
\item Forward-looking difference equation  
\item Agents observe everything
\item Enough conditions to deliver unique stationary solution
\end{itemize}
\item Identification:  Can we identify parameters when we observe \\
\begin{itemize}
\item all the shocks?
\item none of the shocks?
\item some of the shocks?
\item forecasts or asset prices?
\end{itemize}
\end{itemize}
\end{frame}
%\end{comment}
%\end{document}

% --------------------------------------------------------------------------
\begin{frame}
\frametitle{Plan of attack}
\begin{itemize} \itemsep=\bigskipamount
%\item Issue:  using forecasts and bond yields to identify the Taylor Rule
\item Examples  \\
\begin{itemize}
\item [1] Forward-looking difference equation (MA version)
\item [2] Streamlined NK model
\item [3] Forward-looking difference equation (state space version)
\item [4,5,6] Three [??] models of term premiums
\end{itemize}
\item Questions for each  \\
\begin{itemize}
\item What do we observe?  (shocks, state)
\item What can we infer?  (parameters, innovations)
\item Do forecasts or asset prices help?
\end{itemize}
\end{itemize}
\end{frame}
%\end{document}

\section{Example 1: difference equation (moving average version)}
% --------------------------------------------------------------------------
\begin{frame}
\frametitle{Example 1:  forward-looking difference equation}
\begin{itemize} \itemsep=\bigskipamount
\item Model:  moving average representation
\begin{eqnarray*}
    y_t &=& \lambda E_t y_{t+1} + x_t \phantom{xxx} \\
    x_t &=& \alpha(L) w_t, \;\;\;   w_t \sim \mbox{NID}(0,1)  %\phantom{xx}
\end{eqnarray*}
[$|\lambda|<1$, $\alpha$ one-sided and square-summable, $w$ fundamental for $x$]
\item Solution (agents observe everything)
\begin{eqnarray*}
    y_t &=& \beta(L) w_t, \;\;\;
    \beta_j \;=\; \alpha_j + \lambda \alpha_{j+1} + \lambda^2 \alpha_{j+2} \; \cdots
%      \phantom{xxxxxxxx}
\end{eqnarray*}
%\item Example: If $x$ is MA(1), then
%$   \beta_0 = \alpha_0 + \lambda \alpha_{1} $, $ \beta_1 = \alpha_1 $

\item Question:  What do we need to estimate $\lambda$?
%  Notes:
%   * can't run regression, x drives y, not error term
%   * could use forecast, but still need to know x
\end{itemize}
\end{frame}
%\end{document}

% --------------------------------------------------------------------------
\begin{frame}
\frametitle{Example 1A: observable shock}
\begin{itemize}  \itemsep=\bigskipamount
\item $(x,y)$ \\
\begin{itemize}
\item Joint process but degenerate (same innovations)
\item Cross-equation restrictions on parameters ($\beta$ depends on $\lambda,\alpha$)
\end{itemize}
\item Observe variable $y$ (always) \\
\begin{itemize}
\item Estimate $\beta$
\end{itemize}
\item Observe shock $x$ \\
\begin{itemize}
\item Estimate $\alpha$
\item Use connection between $\alpha$ and $\beta$ to infer $\lambda$
\item $\lambda$ overdetermined if $x$ is MA($q$) and $q>1$
\end{itemize}
\end{itemize}
\end{frame}
%\end{document}

% --------------------------------------------------------------------------
\begin{frame}
\frametitle{Example 1B: hidden shock}
\begin{itemize}  \itemsep=\bigskipamount
\item $(x,y)$ still joint process, but we don't observe $x$
\item Observe variable $y$ \\
\begin{itemize}
\item Estimate $\beta$
\item Can't distinguish roles of $\alpha$ and $\lambda$
\end{itemize}
\item Do forecasts help?
\begin{eqnarray*}
    f^k_t \;\;=\;\; E_t y_{t+k} &=& \left[ \beta(L)/L^k\right]_+ w_t  \phantom{xxxx}
\end{eqnarray*}
\hspace{0.4in}[Nope, depends only on $\beta$, can't tell us about $(\lambda,\alpha)$]
%\begin{itemize}
%\item[] Adds no information beyond $y$, which we already know
%\end{itemize}
\end{itemize}
\end{frame}
%\end{document}

% --------------------------------------------------------------------------
\begin{frame}
\frametitle{Example 1C: partly observable shock}
\begin{itemize}  \itemsep=\bigskipamount
\item Model and solution
\begin{eqnarray*}
    y_t &=& \lambda E_t y_{t+1} + x_{1t} + x_{2t} \\
    x_{it} &=& \alpha_{i}(L) w_{it}, \;\;\; (w_{1t},w_{2t}) \sim \mbox{NID}(0,I)
                \phantom{xxx} \\
    \Rightarrow \;\;\;
    y_t &=&  \beta_1(L) w_{1t} + \beta_2(L) w_{2t}
\end{eqnarray*}
\item Observe variable $y$ and shock $x_1$ --- but not $x_2$ \\
\begin{itemize}
\item Estimate $\alpha_{1}$ and $\beta_{1}$
\item Use connection between $\alpha_1$ and $\beta_1$ to infer $\lambda$
\end{itemize}
\item Do forecasts help? [?? logic? order?] \\
\begin{itemize}
\item They help to identify the $w_i$'s
\item We can also recover $x_2$ from the difference equation 
\end{itemize}
\end{itemize}
\end{frame}
%\end{document}

\section{Example 2:  NK model}
% --------------------------------------------------------------------------
\begin{frame}
\frametitle{Example 2: Streamlined NK model}
\begin{itemize} \itemsep=\bigskipamount
\item Model ($p$ = inflation, $i$ = interest rate) 
\begin{align*}
    i_t &\;\;=\;\;  E_t p_{t+1} + x_{1t} \tag{Euler equation} \\
    i_t &\;\;=\;\;  \tau p_{t} +  x_{2t} \tag{Taylor rule} \\
    x_{it} &\;\;=\;\; \alpha_i(L) w_{it}, \;\;\; (w_{1t},w_{2t}) \sim \mbox{NID}(0,I)
            \phantom{xx}
\end{align*}
\item Difference equation and solution
\begin{eqnarray*}
    p_t &=&  \tau^{-1} E_t p_{t+1} + \tau^{-1} (x_{1t} - x_{2t})
        \phantom{\sum} \\
%    \phantom{x}
    \Rightarrow \;\;\;
    p_t &=& \beta_1(L) w_{1t} + \beta_2(L) w_{2t} \\
    i_t &=&  [\beta_1(L)/L]_+ w_{1t} + [\beta_2(L)/L]_+ w_{2t} + x_{1t}
             \phantom{xx}
\end{eqnarray*}
\item Question:  What do we need to estimate $\lambda = \tau^{-1}$?
\end{itemize}
\end{frame}
%\end{document}

% --------------------------------------------------------------------------
\begin{frame}
\frametitle{Example 2A: observable shocks}
\begin{itemize}  \itemsep=\bigskipamount
\item $(x_1,x_2,p,i)$ \\
\begin{itemize}
\item Joint process but degenerate (two-dimensional innovation)
\item Cross-equation restrictions on parameters
\end{itemize}
\item Observe shocks $(x_1,x_2)$ \\
\begin{itemize}
\item Estimate $\alpha_i$'s
\end{itemize}
\item Observe variables $(p,i)$ \\
\begin{itemize}
\item Estimate $\beta_i$'s
\item Use connections between $\alpha_i$ and $\beta_i$ to infer $\lambda = \tau^{-1}$
\end{itemize}
\end{itemize}
\end{frame}
%\end{document}

% --------------------------------------------------------------------------
\begin{frame}
\frametitle{Example 2B: hidden shock (Cochrane variant)}
\begin{itemize}  \itemsep=\bigskipamount
\item Shocks:  $x_1 = 0$, $x_2$ hidden
\item Observe variable $p$  \\
\begin{itemize}
\item Estimate $\beta_2$  (possible because shock is one dimensional)
\item Can't distinguish roles of $\alpha_2$ and $\lambda$
\end{itemize}
\item Does $i$ help?  \\
\begin{itemize}
\item No, contains no information not in $p$
\end{itemize}
\item Do forecasts help? \\
\begin{itemize}
\item No, they add no information beyond $p$
\end{itemize}
\end{itemize}
\end{frame}
%\end{document}

% --------------------------------------------------------------------------
\begin{frame}
\frametitle{Example 2C: partly observable shock (Gertler variant)}
\begin{itemize}  \itemsep=\bigskipamount
\item Observe variables $(p,i)$ and shock $x_1$ \\
\begin{itemize}
\item Estimate $\alpha_{1}$ and $\beta_{1}$
\item Use connection between $\alpha_1$ and $\beta_1$ to infer $\lambda$
\end{itemize}
\item Key insight \\
\begin{itemize}
\item $i$ gives us an additional observable when $x_1 \neq 0$
%\item Allows us to identify $\lambda$
\end{itemize}
\item How might we observe $x_1$?  \\
\begin{itemize}
\item Observe $E_t p_{t+1}$ directly ... or derive from VAR in $(p,i)$
\item Then back $x_1$ out of Euler equation 
\item ?? make sure this lines up with later stuff 
\end{itemize}
\end{itemize}
\end{frame}
%\end{document}

\section{Example 3:  difference equation (state space version)}
% --------------------------------------------------------------------------
\begin{frame}
\frametitle{Example 3: forward-looking difference equation}
\begin{itemize} \itemsep=\bigskipamount
\item Model:  state space representation
\begin{eqnarray*}
    y_t &=& \lambda E_t y_{t+1} + e^\top x_t \phantom{xxxxxx} \\
    x_{t+1} &=& A x_{t} + C w_{t+1},
            \;\;\;\{ w_t \} \sim \mbox{NID}(0,I)
\end{eqnarray*}
[$|\lambda|<1$, $A$ stable \& ``regular'']
\item Solution (agents observe everything)
\begin{eqnarray*}
    y_t &=&  e^\top (I-\lambda A)^{-1} x_t \;\;=\;\; \beta^\top x_t
\end{eqnarray*}
%\smallskip
\item Question:  What do we need to estimate $\lambda$?
\end{itemize}
\end{frame}
%\end{document}

% --------------------------------------------------------------------------
\begin{frame}
\frametitle{Example 3A: observable state and shock}
\begin{itemize}  \itemsep=\bigskipamount
\item Observe state $x$ and vector $e$ (hence shock $e^\top x$)\\
\begin{itemize}
\item Estimate $A$ \;\; [$C$, too, but it's not needed here]
\end{itemize}
\item Observe variable $y$ \\
\begin{itemize}
\item Estimate $\beta$, use $A$ to infer $\lambda$
\begin{eqnarray*}
    \lambda \beta^\top &=& (\beta - e)^\top A^{-1}
     \;\;\; (\lambda \mbox { is overdetermined)}
\end{eqnarray*}

\end{itemize}
\item Forecasts redundant
\begin{eqnarray*}
    f^k_t \;\;=\;\; E_t y_{t+k} &=& \beta^\top A^k x_t \phantom{xxxxx}
\end{eqnarray*}
\end{itemize}
\end{frame}
%\end{document}

% --------------------------------------------------------------------------
\begin{frame}
\frametitle{Example 3: digression on hidden shock/state}
\begin{itemize}  \itemsep=\bigskipamount
\item What's hidden, state $x$ or shock $e^\top x$?
\item Suggestion:  we see the state, not the shock  \\
\begin{itemize} \itemsep=0.25\bigskipamount
\item Lots of econometric approaches to identifying the state \\
%(Sargent \& Sims, Stock \& Watson, Onatski)
\item Another is to build it from forecasts of $y$
\begin{eqnarray*}
    f_t &=& [ y_t, f^1_t, \cdots, f^{n-1}_t ]^\top
            \phantom{xxxxxx}
\end{eqnarray*}
$f_t$ spans state if $A$ is regular and $n = \dim(x)$
\item This leads to transformed state-space equations
\begin{eqnarray*}
    f_t &=&  F x_t  \\
    \;\;\; \Rightarrow
    f_{t+1} &=& F^{-1} A F f_t + F^{-1} C w_{t+1}
        \;\;=\;\; \widehat{A} f_{t} + \widehat{C} w_{t+1} \\
    y_t &=& e^\top F^{-1} (I-\lambda \widehat{A})^{-1} f_t
            \;\;=\;\;
            \widehat{e}^\top (I-\lambda \widehat{A})^{-1} f_t \;\;=\;\; \widehat{\beta}^{\top} f_t
            \phantom{xxxxxxxxxx}
\end{eqnarray*}
\end{itemize}
%\item From here on:  assume we observe $x$
\end{itemize}
\end{frame}
%\end{document}

% --------------------------------------------------------------------------
\begin{frame}
\frametitle{Example 3B: hidden shock}
\begin{itemize}  \itemsep=\bigskipamount
\item Shock $e^\top x$ not observed ($e$ not known)
\item Observe variable $y$, state $x$ \\
\begin{itemize}
\item From $y$:  estimate $\beta$
\item From $x$:  estimate $A$
\item But since we don't know $e$, can't infer $\lambda$
\end{itemize}
\item Do forecasts help? \\
\begin{itemize}
\item They span the state, not otherwise helpful
\end{itemize}
\end{itemize}
\end{frame}
%\end{document}

% --------------------------------------------------------------------------
\begin{frame}
\frametitle{Example 3C: partly observable shock}
\begin{itemize}  \itemsep=\bigskipamount
\item Model
\begin{eqnarray*}
    y_t &=& \lambda E_t y_{t+1} + e_1^\top x_{t} + e_2^\top x_{t}
    \\
   \Rightarrow  \;\;\;
    y_t &=&  (e_1+e_2)^\top (I-\lambda A)^{-1} x_t \;\;=\;\; \beta^\top x_t
    \phantom{xxxxx}
\end{eqnarray*}
\item Observe variable $y$, state $x$ \\
\begin{itemize}
\item From $y$:  estimate $\beta$
\item From $x$:  estimate $A$
\end{itemize}
\item Is it enough to observe shock $e_1^\top x$?  \\
\begin{itemize}
\item Can't easily disentangle effects of $e_1$ and $e_2$ \\
\begin{itemize}
\item we know $\beta$ ($n$-dimensional, say)
\item we want to infer $(e_2,\lambda)$ ($n+1$-dimensional)
\item $\Rightarrow$ we need one restriction
\end{itemize}
\end{itemize}
\end{itemize}
\end{frame}
%\end{document}

% --------------------------------------------------------------------------
\begin{frame}
\frametitle{Example 4: Streamlined NK model (state space version)}
\begin{itemize} \itemsep=\bigskipamount
\item Model
\begin{align*}
    i_t &\;\;=\;\;  - \log E_t \left( m_{t+1} / p_{t+1} \right)  \tag{Euler equation} \\
    i_t &\;\;=\;\;  r + \tau p_{t} +  e_2^\top x_{t} \tag{Taylor rule} \\
    \log m_{t+1} &\;\;=\;\; - \delta - \alpha \log g_{t+1} \\
    \log g_{t+1} &\;\;=\;\; g + e_1^\top x_t \\
    x_{t+1} &\;\;=\;\; A x_{t} + C w_{t+1},
            \;\;\;\{ w_t \} \sim \mbox{NID}(0,I)
\end{align*}
\item Solution:  $ p_t = \beta^\top x_t $ 
\begin{eqnarray*}
    \beta^\top &=& (\alpha e_1 - e_2)^\top (\tau I - A)^{-1} \\
    r &=& \delta + \alpha g - (\alpha e_1 + \beta)^\top C C^\top (\alpha e_1 + \beta)/2 
\end{eqnarray*}

\item Question:  What do we need to estimate $\lambda = \tau^{-1}$?
\end{itemize}
\end{frame}
%\end{document}

% --------------------------------------------------------------------------
\begin{frame}
\frametitle{Example 4: bond pricing model}
\begin{itemize}  \itemsep=\bigskipamount
\item Observe state $x$ and vectors $(e_1,e_2)$\\
\begin{itemize}
\item Estimate $A$ \;\; [$C$, too, if we like]
\end{itemize}
\item Observe variable $y$ \\
\begin{itemize}
\item Estimate $\beta$, use $A$ to infer $\lambda$
\end{itemize}
\item Forecasts redundant
\begin{eqnarray*}
    f^k_t \;\;=\;\; E_t y_{t+k} &=& \beta^\top A^k x_t \phantom{xxxx}
\end{eqnarray*}
\end{itemize}
\end{frame}
\end{document}

% --------------------------------------------------------------------------
\begin{frame}
\frametitle{Example 3B: Cochrane variant}
\begin{itemize}  %\itemsep=\bigskipamount
\item Observe $x$, not $e_2$, and $e_1= 0$
\item Observe variable $y$ \\
\begin{itemize}
\item We can't estimate either $\beta$ or its ingredients $(A,\lambda)$
\end{itemize}
\item Do forecasts help? \\
\begin{itemize}
\item They span the state
\begin{eqnarray*}
    f_t &=& [ y_t, f^1_t, \cdots, f^{n-1}_t ]^\top
            \;\;=\;\; F x_t \phantom{\sum_K} \\
\Rightarrow \;\;
    f_t &=& F^{-1} A F f_t + F^{-1} C w_{t}
        \;\;=\;\; \widehat{A} f_{t-1} + \widehat{C} w_{t} \\
    y_t &=& e^\top F^{-1} (I-\lambda \widehat{A})^{-1} f_t
            \;\;=\;\;
            \widehat{e}^\top (I-\lambda \widehat{A})^{-1} f_t \;\;=\;\; \widehat{\beta}^{\top} f_t
\end{eqnarray*}
\end{itemize}
\begin{itemize}
\item $f_t$ spans state if $A$ is regular
\item Allows us to estimate $\widehat{A}$ and $\widehat{\beta}$
\item But unless we know $\widehat{e}$ we're stuck  [??]
\end{itemize}
\end{itemize}
\end{frame}
%\end{document}

% --------------------------------------------------------------------------
\begin{frame}
\frametitle{Example 2C: partly observable shock}
\begin{itemize}  \itemsep=0.75\bigskipamount
\item Model
\begin{eqnarray*}
    y_t &=& ??  %\lambda E_t y_{t+1} + e_1^\top x_{t} + e_1^\top x_{t}
\end{eqnarray*}
\item Observe shock $e_1^\top x$ \\
\begin{itemize}
\item Estimate $\alpha_{1}$
\end{itemize}
\item Observe variable $y$ \\
\begin{itemize}
\item Estimate $\beta_{1}$, use $\alpha_1$ to infer $\lambda$
\end{itemize}
\item Questions \\
\begin{itemize}
\item Can we recover $x_2$?  Yes:  back out of difference equation
\item Do forecasts help?  Nope, they're redundant here
\end{itemize}
\end{itemize}
\end{frame}
%\end{document}
\end{document}

% --------------------------------------------------------------------------
% --------------------------------------------------------------------------
NYU, March 2012  

This paper started with a conversation Stan and I had with Mark several years ago, 
but I'll set that aside for a moment.  

The issue is how you identify the Taylor rule: if the Fed changes its
policy rule, how would you see that in the data?


Mike, Stan, and I :
eg, equity premium.  
Another is the term structure.  


% --------------------------------------------------------------------------
Minnesota, May 2012 
This is an old school paper, which I like, 
but it's also an attempt to make up for missing out on Minn in the 1970s. 

When we were graduate students at Yale, 
Tim and I were sitting around reading Samuelson's Foundations...  

Now what we didn't realize at the time was that the 1970s was a great 
time to study economics --- at Minnesota.  
With Patrick's help...  We read papers by Sargent, Wallce, Sims, and their students 
and tried to get up to speed.  
A little later, we met the authors.  And we said to ourselves, that's it?  

This paper...  
The idea is to use

We build on work by Lars, Tom, and Chris. 
Mike's guess is that Lars will tell us he knew this in 1979, 
and he probably did, 
that Chris will say it's wrong, 
and that Tom will agree with both of them.  

