\documentclass[11pt]{article}

\RequirePackage{comment}
\RequirePackage{graphicx}
\RequirePackage[hypertex]{hyperref}
\usepackage{amssymb}
\usepackage{amsfonts}
\usepackage{booktabs}

\newcounter{fig}
\newcounter{tab}

\oddsidemargin=0.250truein
\evensidemargin=0.250truein
\textwidth=6.0truein
\topmargin=-0.250truein
\textheight=8.25truein

\renewcommand{\thefootnote}{\fnsymbol{footnote}}
\newcommand{\N}{\mathcal{N}}
\renewcommand{\epsilon}{\varepsilon}
\newcommand{\var}{\mbox{\it Var\/}}
\newcommand{\phm}{\phantom{--}}
%\newcommand{\hs}{\hspace{5pt}}

% next three lines for biblio setup
\newcommand{\paperstart}[0]{\par\penalty 1 \hangindent=32pt \hangafter=1}
\newcommand{\paperend}[0]{\penalty -1 \par\vskip 0pt plus 0pt minus 0pt}
%\newcommand{\paperend}[0]{\penalty -1 \par\vskip 15pt plus 25pt minus 10pt}
\newcommand{\paper}[1]{\paperstart #1 \paperend}

\newcommand{\Qm}{\mathbb{P}^*}
\newcommand{\Pm}{\mathbb{P}}

% Example counter and macro
\newcounter{remark}
\setcounter{remark}{1}
\newcommand{\remark}[1]{%
    \noindent{\it Remark \arabic{remark} (#1).\/}%
    \addtocounter{remark}{1}
    }

\begin{document}
\thispagestyle{empty}
\newlength{\oldparindent}
\oldparindent=\parindent
\parindent=0.0in

\vspace*{1.0in}  %% 11pt
%  ** use LARGE with 11pt, Large with 12pt
\noindent{\LARGE\bf Identifying macroeconomic parameters with} \vspace*{0.07in}\\
%\vspace{0.001in} \\
\noindent{\LARGE\bf forecasts, asset prices, and structure\footnote{%
Preliminary and incomplete:  no guarantees of accuracy or sense.
We welcome comments, including references to related papers we
inadvertently overlooked.
This started with our reading of John Cochrane's paper on the same subject
and subsequent emails and conversations with him and Mark Gertler.
We thank them both.
We also owe an obvious debt to Lars Hansen, Thomas Sargent,
and Christopher Sims.  
% ??
%The latest version of the paper is available at:
%\newline\centerline
%{\url{http://pages.stern.nyu.edu/~dbackus/GE_asset_pricing/disasters/BCM_disasters_latest.pdf}.}
}}

\vspace{0.5in}
\noindent{%
David Backus,\footnote{
        Stern School of Business, New York University, and NBER;
        dbackus@stern.nyu.edu.}
Mikhail Chernov,\footnote{
        London Business School, London School of Economics, and CEPR;
        mchernov@london.edu.}
and Stanley Zin\footnote{
        Stern School of Business, New York University, and NBER;
        stan.zin@nyu.edu.}
}

\vspace{0.4in}
{\today} \\ Preliminary and incomplete
%\\ Warning:  second half is a mess


\vfill
{
%\vspace{1.8in}
\noindent{\bf Abstract}

\medskip
\noindent{%
Identification problems arise naturally in forward-looking
models when agents observe more than econometricians.
One approach is additional information,
including economic forecasts and asset prices. 
Another is tighter economic structure:
what is sometimes called cross-equation restrictions.  
We show how each aids in the identification of structural 
parameters, including 
the inflation parameter of a Taylor rule.
As a rule, asset prices help to identify the state, 
and cross-equation restrictions help to identify structural parameters.
All of this is done with variations on a single example.  

\bigskip
%\medskip
\noindent{\bf JEL Classification Codes: }  E43, E52, G12.

\medskip
{\noindent{\bf Keywords: } forward-looking models; information sets;
forecasts; forward rates; monetary policy.}
\smallskip
}

%\end{document}
% Start of text ***************************************************************
\newpage
\parskip=\bigskipamount
%\parindent=\oldparindent
\setcounter{footnote}{0}
\renewcommand{\thefootnote}{\arabic{footnote}}
\setcounter{page}{1} \thispagestyle{empty}
\newlength{\oldbaselineskip}
\setlength{\oldbaselineskip}{\baselineskip}
\parskip= \bigskipamount

\section{Introduction}

We explore the possibility of using forecasts, asset prices, 
and cross-equation restrictions 
to identify structural parameters of macroeconomic models. 
In stationary Markov models, 
macroeconomic quantities and asset prices are functions of 
the state, 
so we might consider using a combination of the two to reverse
the operation and infer the state from the values.  
In parsimonious models, we might also expect the additional information
from asset prices to ....  


... This unified field theory of macro-finance is far from perfect, 
but we don't see identification as its central problem.  

We work through some examples that illustrate the central issues
governing identification in forward-looking models,
including the potential role played by forward rates.
We wouldn't call the example simple,
but it's the simplest we've been able to come up with.
Even here some subtle issues arise.
The central issue is what agents and econometricians observe.
When econometricians (namely, us) observe less than agents,
there are inevitably difficulties with identification.
The logic follows Hansen and Sargent (1980, 1991)
and related work by them and others.
Forward rates can be helpful here, since they
provide the econometrician with additional information.
Whether this information is enough to identify the model's parameters
depends on the model.

Model:  typically expressed as a forward-looking difference equation \\
Solution:  state follows a stationary process, and 
prices and quantities are functions of the state \\
Identification concerns the estimation of parameters

Punchline...  


\section{Forward-looking models 1:  moving average version}



We'll start with this one-dimensional example of a forward-looking
rational expectations model:
\begin{eqnarray}
    y_t &=& \lambda E_t y_{t+1} + x_t .
    \label{eq:example1}
\end{eqnarray}
Here $y$ is an endogenous forward-looking variable,
$x$ is an exogenous shock or forcing process,
and $\lambda$ is a parameter whose absolute value is less than one.
The shock $x$ is an arbitrary scalar moving average,
\begin{eqnarray*}
    x_t &=& \sum_{j=0}^\infty \alpha_j w_{t-j} \;\;=\;\; \alpha(L) w_t ,
\end{eqnarray*}
where $x$ is stationary (square summable) and
$ \{ w_t \} \sim \mbox{NID}(0,1) $.
Under these conditions, $y$ has a unique stationary solution.
In the scalar moving average case, the solution has the form
$  y_t = \beta(L) w_t  $,
with
\begin{eqnarray*}
    \beta_j &=& \sum_{i=0}^\infty \lambda^i \alpha_{j+i}
\end{eqnarray*}
for $j\geq 0$.
See Appendix \ref{app:hs-formulas}.

Our mission is to estimate the parameters,
particularly $\lambda$.
Identification depends in large part on what we observe.
We assume throughout that the agents in this economy
observe $x$ and $y$.
Indeed, without this information, our solution
of (\ref{eq:example1}) wouldn't make sense.
The question is what econometricians observe.
We assume we always observe $y$ and consider
the impact of observing or not observing $x$.
The question is whether we, the econometricians,
can estimate the parameters anyway.
We illustrate the argument with a first-order
moving average or MA(1) for $x$:
\begin{eqnarray*}
    x_t &=& \alpha_0 w_t + \alpha_1 w_{t-1}
        \;\;=\;\; \alpha_{0} (w_t + \theta w_{t-1}) .
\end{eqnarray*}
Then $y$ is also MA(1),
with parameters $\beta_0 = \alpha_0 + \lambda \alpha_1$ and $\beta_1 = \alpha_1$.

In this setting, identification is basically a matter of
matching up parameters with properties of observables.
We assume we know the autocovariance function $\gamma_y(k)$ of $y$,
where $\gamma_y(k) \equiv E (y_t y_{t-k}) $.
If $x$ is observed, we also know the autocovariance  function
$\gamma_x(k)$ for $x$.

\subsection{What do we observe?}


\subsection{Example:  streamlined New Keynesian model}

The simplest example in Cochrane (2007) consists of two equations:
\begin{eqnarray}
    i_t &=& r + E_t p_{t+1} + x_{1t}
        \label{eq:cochrane-euler} \\
    i_t &=& r + \tau p_t + x_{2t} ,
        \label{eq:cochrane-taylor}
\end{eqnarray}
The variables are the nominal interest rate $i$ and inflation $p$.
Think of the first equation as a simple version of an Euler equation (EE)
and the second as a Taylor rule (TR).
Let the shocks $x_{it}$ be independent moving averages $\alpha_i(L)w_{it}$.
We'll set the real interest rate $r = 0 $ for now.

[Should we switch to vector version, which imposes invertibility on the
overall system but not on subsets?]

{\it Solution.}
Equations (\ref{eq:cochrane-euler}) and (\ref{eq:cochrane-taylor})
give us the stochastic difference equation
\begin{eqnarray}
    E_t p_{t+1} &=& \tau p_t - x_{1t} + x_{2t} .
        \label{eq:cochrane-diff}
\end{eqnarray}
We're looking for a stationary solution for $p$.
If $|\tau |>1 $, there's a unique stationary solution of the form
$ p_t = \pi_1(L) w_{1t} + \pi_2(L) w_{2t}$.
Thus:  given parameters $\{ \tau,\alpha_{ij} \}$ we
can derive the coefficients $\{ \pi_{ij} \}$ of the inflation process.

{\it Identification.}
Suppose we observe inflation ($p$) and the interest rate ($i$),
but not the shocks ($x_1$ and $x_2$).
Can we identify the Taylor rule parameter ($\tau$)?
The difficulty is that the behavior of $p$ and $i$ combines
the behavior of the shocks and the TR parameter,
and it's not clear we can disentangle them.
Here are some examples.

\begin{itemize}
\item Special case 1 ($x_2=0$).
Identification follows directly from estimating the
Taylor rule (\ref{eq:cochrane-taylor}),
which is exact in this case.
Both the lhs and rhs are observable.

A little algebra shows how this works.
Since the model has a univariate shock,
we can estimate $ \pi_1(L)$ from a long enough time series for $p$.
From this process, we can infer the coefficients for expected inflation
($[\pi_1(L)/L]_+$ in Hansen-Sargent notation).
The interest rate and the Euler equation then tell us $x_1$
and its coefficients $\alpha_1(L)$.
The coefficients of the lhs ($[\pi_1(L)/L]_+ + \alpha_1(L)$) are
$\tau$ times those of the rhs ($\pi_1(L)$):
\begin{eqnarray*}
    w_{t-j}:  &&  \pi_{1j+1} + \alpha_{1j} \;=\;
                \tau^j \sum_{i=j+1}^{\infty} \tau^{-i} \alpha_{1i}
                    + \alpha_{1j}
                \;=\; \tau \pi_{1j} .
\end{eqnarray*}
[This holds by construction; verification here simply
tells us we did the calculations right.]


\item Special case 2 ($x_1=0$).
This is the example Cochrane looks at.
Again, we observe $p$ and $i$.
The former gives us the inflation coefficients $\pi_2$.
The latter gives us no new information,
since with $x_1=0$
expected inflation is implied by the inflation process.
The result is that we can't isolate the roles of $p_t$ and $x_{2t}$
in the Taylor rule (\ref{eq:cochrane-taylor}), hence can't estimate $\tau$.

Here's an example.
Suppose $x_2$ is MA(1).  Then $p$ is MA(1), too.
The inflation coefficients are
\begin{eqnarray*}
    \pi_{20} &=& - \tau^{-1} (\alpha_{20} + \tau^{-1} \alpha_{21} ) \\
    \pi_{21} &=& - \tau^{-1} \alpha_{21} .
\end{eqnarray*}
We can estimate the $\pi$'s from inflation data,
but we can't disentangle the impact
of the shock (the $\alpha$'s) from the policy parameter ($\tau$).
That's true even if $x_2$ is white noise ($\alpha_{2j} = 0$ for $j \geq 1$):
the inflation coefficient is $\pi_{20} = - \tau^{-1} \alpha_{20}$ and we
can't separate the two components.
Alternatively, let $x_2$ be AR(1), so that
$ \alpha_{2j} = \varphi^j \alpha_{20} $.
In this case, $ \pi_{2j} = - \varphi^j \alpha_{20} /(\tau-\varphi)$,
so inflation is AR(1) with the same autoregressive parameter.
Inflation and the shock are perfectly correlated, so
there's no way we can disentangle their effects in the Taylor rule.
An inflation process can be reconciled with any choice of $\tau$ we
like by adjusting $\alpha_{20}$.



\item Shocks in both equations.
One of the lessons here is getting extra information out of the
interest rate (special case 1).
If $x_1 = 0$ we can't do that.
Another is that the TR shock $x_2$ shows up in $p$,
which makes it difficult to separate their effects in the TR
(special case 2).
The question is how far we can go if we have shocks
in both equations.

Gertler's example shows how we might handle two shocks.
Let the two state variables be AR(1):
\begin{eqnarray*}
    x_{it} &=& \varphi_i x_{it-1} + w_{it}  .
\end{eqnarray*}
with  $\varphi_2 = 0$
(we can generalize this later on).
In either case, the state space is essentially $ (x_1,x_2)$,
so we  drop the infinite MA notation.
The inflation process has the form
$ p_t = \pi_1 x_{1t} + \pi_2 x_{2t} $.
Substituting into (\ref{eq:cochrane-diff}) and collecting terms gives us
$ \pi_1 = 1/(\tau-\varphi_1) $ and $ \pi_2 = -1/\tau $.
Observables are therefore
\begin{eqnarray*}
    p_t  &=& [1/(\tau-\varphi_1)] x_{1t}- (1/\tau) x_{2t}  \\
%   E_t p_{t+1}  &=& [\varphi_1/(\tau-\varphi_1)] x_{1t}  \\
        i_t &=& [\tau/(\tau-\varphi_1)] x_{1t} .
\end{eqnarray*}
%[Expected inflation is indirectly observable from regressing
%inflation on past states, here spanned by lags of $p$ and $i$.]
We can estimate $\varphi_1$ from the autocorrelation of $i$
and $\tau$ from the ratio of the interest rate to expected inflation.



\section{Forward-looking models 2:  state space version}

Here's a more formal approach to the same problem.
The idea is to write the dynamics of the observables in
terms of the dynamics of the shocks.
Here if the shocks are a VAR, then so are the observables.
The VAR for the observables can be estimated,
so the question is whether we can use its estimated coefficients
to uncover the TR parameter $\tau$.
Suppose the shocks follow
\begin{eqnarray*}
    x_{t+1}  &=&  A x_t + B w_{t+1} ,
\end{eqnarray*}
with $x = (x_1,x_2)$ and $w$ vector white noise.
In our example, $ A = [\varphi_1, 0; 0, 0]$.
Then the observables $ y = (p,i)$ are a
linear transformation of $x$:
$ y = C x $.
In our example,
\begin{eqnarray*}
   C  &=&
    \left[
    \begin{array}{cc}
         1/(\tau-\varphi_1) & - 1/\tau \\ \tau/(\tau-\varphi_1) & 0
    \end{array}
    \right] .
\end{eqnarray*}
Let's assume that $C$ is invertible.
Then in terms of observables, we also have a VAR, namely
\begin{eqnarray*}
    y_{t+1}  &=&  C A C^{-1} y_t + C B w_{t+1}
             \;=\;  A^* y_t + B^* w_{t+1}
\end{eqnarray*}
Now to identification.
We can estimate the autoregressive parameters $A^*$ of the observables;
the question is whether we can deduce $\tau$ from them.
In the example,
\begin{eqnarray*}
   A^*  &=&
    \left[
    \begin{array}{cc}
        0 & \varphi_1/\tau  \\  0 & \varphi_1
    \end{array}
    \right] ,
\end{eqnarray*}
so we can compute $\tau = a^*_{22}/a^*_{12}$.
[This is tedious; we did it with Matlab's Maple toolbox.]
\begin{comment}
Even if $\varphi_2 \neq \varphi_1$ we can identify $\tau$ from estimates
of $A^*$:
\begin{eqnarray*}
   A^*  &=&
    \left[
    \begin{array}{cc}
        \varphi_2 & (\varphi_1-\varphi_2)/\tau  \\  0 & \varphi_1
    \end{array}
    \right] .
\end{eqnarray*}
[Guess:  you need $A$ to be triangular.]
\end{comment}

This line of thought can easily be generalized,
although the expressions can get complicated.
Let the expectational difference equation be
\begin{eqnarray*}
    E_t p_{t+1} &=& \tau p_t - u_1^\top x_{t} + u_2^\top x_{t} ,
%        \label{eq:var-diff}
\end{eqnarray*}
where $u_1$ and $u_2$ are known vectors.
(In our example, $u_i$ picks out the $i$th element of $x$.)
We guess $ p_t = a^\top x_t$ and derive
\begin{eqnarray}
    a^\top &=& (u_1 - u_2)^\top (\tau I - A)^{-1} .
\end{eqnarray}
%[$A^\top$?]
The observables are then
\begin{eqnarray*}
   y_t  &=&
    \left[
        \begin{array}{c}
         p_t \\ i_t
        \end{array}
    \right]
        \;=\;
    \left[
    \begin{array}{c}
         a^\top \\  a^\top A + u_1^\top
    \end{array}
    \right]  x_t
        \;=\;  C x_t.
\end{eqnarray*}
We estimate $A^*$ and then ask whether we can recover $\tau$.
By counting, you might guess that this won't work without some
restrictions on $A$:
we estimate 4 elements of $A^*$, which isn't enough to
nail down the 4 elements of $A$ plus $\tau$.
Clearly it works, as described, if all the elements of $A$
but the first one are zero.
%It also seems to work if we make only $a_{22}$ nonzero.
Can we go beyond that?
Suppose $A$ is diagonal.
This is horribly nonlinear,
but the diagonal elements of $A$ are the eigenvalues of $A^*$.
[Remember:  $A$ and $A^*$ are similar.]
From there, we can find $\tau$.
For example [this courtesy of Matlab] the upper right element is
\begin{eqnarray*}
    a^*_{12} &=&  \frac{a_{11}-a_{22}}{\tau - a_{22}} .
\end{eqnarray*}
As long as the eigenvalues aren't equal, we can find $\tau$.
[Problem:  we don't know which eigenvalue is which
element of $A$, so we get two possible estimates of $\tau$.]

Could we make $A$ triangular?  Not clear.
Also not clear whether we could use information on covariances:
eg, assume $B$ diagonal.

\end{itemize}

Extensions:
(i) VAR(2), (ii) bond yields (EH), (iii) larger vector of x's.


\section{Term structure models}


\subsection{Essential Affine}

Which model?  Essential?  Or skip this?
Suggestion:  start with model in which ``P and Q'' versions of A are different.
What difficulties does that cause us?


\subsection{A macro model}

One of Stan's models...


\section{Outstanding issues}

Spanning ....

Errors in observations



\section{Conclusions}


% ***********************************************************************
\pagebreak
\baselineskip=\oldbaselineskip
\appendix
\section{Solving expectational difference equations}
\label{app:hs-formulas}

{\it Scalar moving average shock process.\/}
Here's a useful result from Hansen and Sargent (1980, section 2)
and Sargent (1987, section XI.19).
An expectational difference equation with stationary forcing
variable $x$ generates a
``geometric distributed lead'':
\begin{eqnarray*}
    y_t &=&    \lambda E_t y_{t+1} + x_t  \\
        &=&  \lambda E_t (\lambda  E_{t+1} y_{t+2} + x_{t+1}) +  x_t \\
        &=&  \sum_{j=0}^\infty \lambda^j  E_t x_{t+j} .
\end{eqnarray*}
If $ x_t = \sum_{j=0}^\infty \alpha_j w_{t-j} = \alpha(L) w_t $,
with $w$ white noise, then what is $y_t$?
A unique stationary solution $ y_t = \beta(L) w_t $
exists if $x$ is stationary and $ |\lambda|<1 $,
but what is $\beta(L)$?

Conditional expectations of $x$ have the form
\begin{eqnarray*}
    E_t x_{t+j}  &=&  [\alpha(L)/L^j]_+ w_t
                 \;=\;  \sum_{i=0}^\infty \alpha_{j+i} w_{t-i}
\end{eqnarray*}
(The subscript ``+'' means ignore negative powers of $L$.)
Therefore the coefficient of $ w_{t-i}$ in the distributed lead is
\begin{eqnarray*}
    \beta_i &=& \sum_{j=0}^\infty \lambda^j \alpha_{i+j} .
\end{eqnarray*}
This tells us, for example, that if $x$ is MA($q$), then so is $y$:
if $ \alpha_{j} = 0$ for $ j > q $,
then $ \beta_j = 0 $ over the same range.

There's a ``lag polynomial'' version that expresses the result in
compact form.
We're looking for a solution $y_t = \beta(L) w_t$ satisfying the
expectational difference equation:
\begin{eqnarray*}
    \beta(L) w_t &=& [\beta(L)/L]_+ w_t + \alpha(L) w_t  .
\end{eqnarray*}
The solution is
\begin{eqnarray*}
    \beta(L) &=&  \frac{L\alpha(L) - \lambda \alpha(\lambda)}{L-\lambda}
\end{eqnarray*}
See the references mentioned above.

{\it Vector autoregressive shock process.\/}
Here's a related result adapted from
Ljungqvist and Sargent (2005, section 2.4).  [Earlier ref??]
It extends the previous result to higher dimensional processes
that can be expressed as stationary vector autoregressions.
Consider the system
\begin{eqnarray*}
    y_t &=&    \lambda E_t y_{t+1} + e^\top x_{t}  \\
    x_{t+1}  &=&  A x_t + C w_{t+1} ,
\end{eqnarray*}
where
$A$ is stable (eigenvalues less than one in absolute value),
$e$ is an arbitrary vector,
and  $w \sim \mbox{NID}(0,I)$.
The solution in this case is
\begin{eqnarray*}
    y_t &=&  \sum_{j=0}^\infty \lambda^j  e^\top E_t x_{t+j}
        \;=\;  e^\top \sum_{j=0}^\infty \lambda^j  A^j x_{t}
        \;=\;  e^\top (I-\lambda A)^{-1} x_t.
\end{eqnarray*}
The last step follows from the matrix geometric series.

There's a method of undetermined coefficients version of this.
Guess $ y_t = \beta^\top x_t$ for some vector $\beta$
(we know the solution has this form from what we just did).
Then the difference equation tells us
\begin{eqnarray*}
    \beta^\top x_t &=&  \beta^\top \lambda A x_t + e^\top x_t .
\end{eqnarray*}
Collecting terms in $x_t$ gives us
$ \beta^\top =  e^\top (I-\lambda A)^{-1} $, as stated.
What this approach misses is the requirement that $A$ be stable.



%\end{document}

% References ******************************************************************
\pagebreak
\parskip=0.5 \bigskipamount
\parindent=0.0in
\baselineskip=\oldbaselineskip
\section*{References}

%\paper{Ang, Dong, and Piazzesi...}

\paper{Backus, David K., and Stanley E. Zin, 1994,
    ``Reverse engineering the yield curve,''
    NBER Working Paper No. 4676, March.}

%\paper{Bansal, Ravi, and Amir Yaron, 2004,
%    ``Risks for the long run: A potential resolution of asset pricing
%     puzzles,''
%    {\it Journal of Finance\/} 59, 1481-1509.}

\paper{Clarida, Richard, Jordi Gali, and Mark Gertler, 1999,
    ``The science of monetary policy,''
    {\it Journal of Economic Literature\/} 37, 1661-1707.}

%\paper{Cochrane, John H., 2009, JME ...}

\paper{Cochrane, John H., 2011,
    ``Determinacy and identification with Taylor rules,'' 
    {\it Journal of Political Economy\/} 119, 565-615.} 

\paper{Collin-Dufresne, Pierre, and Robert S. Goldstein, 2002,
    ``Do bonds span the fixed income markets? Theory and evidence for
    unspanned stochastic volatility,''
    {\it Journal of Finance\/} 57, 1685-1730.}

\paper{Duffee, Gregory R., 2009,
    ``Information in (and not in) the term structure,''
    manuscript, August.}

%\paper{Epstein, Larry G., and Stanley E. Zin, 1989,
%    ``Substitution, risk aversion, and the temporal behavior of consumption
%    and asset returns: a theoretical framework,''
%    {\it Econometrica\/} 57, 937-969.}

\paper{Gallmeyer, Michael, Burton Hollifield, and Stanley Zin, 2005,
    ``Taylor rules, McCallum rules and the
    term structure of interest rates,''
    {\it Journal of Monetary Economics\/} 52, 921-950.}

\paper{Gallmeyer, Michael F., Burton Hollifield, 
        Francisco Palomino, and Stanley Zin, 2007,
    ``Arbitrage-free bond pricing with dynamic macroeconomic models,''
    {\it Federal Reserve Bank of St Louis Review\/} 89, 305-26.} 
    
\paper{Hansen, Lars Peter, and Thomas J. Sargent, 1980,
    ``Formulating and estimating dynamic linear rational expectations
    models,''
    {\it Journal of Economic Dynamics and Control\/} 2, 7-46.}

\paper{Hansen, Lars Peter, and Thomas J. Sargent, 1991,
    {\it Rational Expectations Econometrics\/},
    San Francisco:  Westview Press.}

%\paper{Hansen, Lars Peter, and Thomas J. Sargent, 2005,
%    {\it Recursive Models of Dynamic Linear Economies\/}
%    manuscript.}

%\paper{Lippi, Marco, and Lucrezia Reichlin, 1994,
%    ``VAR analysis, nonfundamental representations,
%    and Blaschke matrices,''
%    {\it Journal of Econometrics\/} 63, 307-325.}

\paper{Onatski, Alexi, factors, Econometrica, 2011??}

\paper{Piazzesi, Monika, and Martin Schneider, 2011,
    ``Trend and cycle in bond premia,''
    manuscript, January.}

%\paper{Rondina, Giacomo, and Todd B. Walker, 2009,
%    ``Information equilibria in dynamic economies,''
%    manuscript, November.}

\paper{Sargent, Thomas, 1972,
    ``Rational expectations and the term structure
    of interest rates,''
    {\it Journal of Money, Credit and Banking\/} 4, 74-97.}

\paper{Sargent, Thomas J.,  1987,
    {\it Macroeconomic Theory (2nd edition)\/}
    New York:  Academic Press.}

\paper{Sargent and Sims ...}

\paper{Sims...} 

\paper{Smith, Josephine, and John Taylor, 2009,
    ``The term structure of policy rules,''
    {\it Journal of Monetary Economics\/} 56, 907-917.}

\paper{Stock and Watson, factor paper...}

\end{document}

% ****************************************************************************
