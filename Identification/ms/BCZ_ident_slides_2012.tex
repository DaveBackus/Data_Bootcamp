\documentclass{beamer}
%\documentclass[leqno]{beamer}

\mode<presentation>

\usepackage{graphicx}
\usepackage{comment}
\usepackage{booktabs}
\usepackage{amsmath}
\usepackage{hyperref}

%\hypersetup{
%    colorlinks=true,        % kills boxes
%    allcolors=blue,
%    pdftitle={Identification},
%    pdfauthor={Backus, Chernov, \& Zin},
%    pdfcreator={Dave Backus @ NYU},
%    pdfstartview={FitH},
%    pdfpagemode={UseNone},
%    pdfnewwindow=true,      % links in new window
%    linkcolor=blue,         % color of internal links
%    citecolor=blue,         % color of links to bibliography
%    filecolor=blue,         % color of file links
%    urlcolor=blue           % color of external links
% see:  http://www.tug.org/applications/hyperref/manual.html
%}
%\usetheme{Frankfurt}
%\usetheme{Warsaw}
\usetheme[secheader]{Boadilla}

%\setbeamercovered{transparent}
%\setbeameroption{show notes}
\setbeameroption{hide notes}

\special{landscape}

% Basic input for footer
\title[Identification]{}
\author[Backus, Chernov, \& Zin]{}
\institute[NYU, LSE]{}
\date[]{\today}

%****************************************************************************
\begin{document}
\vspace*{0.85in}
\centerline{\Large\bf Identification of Taylor Rules}
%\vspace*{0.05in}
%\centerline{\Large\bf In Macro-Finance Models}

\bigskip\bigskip\bigskip\medskip
\centerline{David Backus, Mikhail Chernov,}
\centerline{and Stanley Zin}

\bigskip\medskip\medskip\medskip
\centerline{NYU Macro Lunch $|$ March 8, 2012}

\vfill
{\tiny \hspace{7pt}This version: \today}
{\hfill \includegraphics[scale=0.25]{stern_logo3.pdf}}

%\end{document}
%\begin{comment}
% --------------------------------------------------------------------------
\begin{frame}
\frametitle{Identification}
\begin{itemize} \itemsep=\bigskipamount
\item A model is an interpretation of the data
\item A model is identified if the interpretation is unique
\end{itemize}
\end{frame}
%\end{document}

% --------------------------------------------------------------------------
\begin{frame}
\frametitle{The problem}
\begin{itemize} \itemsep=0.5\bigskipamount
\item Model:  forward-looking difference equation
\begin{eqnarray*}
    y_t &=& \lambda E_t y_{t+1} + x_{1t} + x_{2t} \phantom{xxx}
\end{eqnarray*}
\item Questions \\
\begin{itemize}
\item How do we identify $\lambda$?
\item Do we need to observe the shocks $(x_1,x_2)$?
\item Are forecasts and asset prices helpful? Cross-equation restrictions?
%\item Connection to methods that identify shocks?
\end{itemize}
\item Answers \\
\begin{itemize}
\item We need some combination of observables and structure
\item We need to observe something, but not everything
\item Yes and yes %Forecasts, asset prices, and economic structure all help
%\item Identifying $\lambda$ and $x$ closely related
\end{itemize}
\end{itemize}
\end{frame}
%\end{document}

\begin{comment}
% --------------------------------------------------------------------------
\begin{frame}
\frametitle{Rules of the game}
\begin{itemize} \itemsep=\bigskipamount
\item Model \\
\begin{itemize}
\item Forward-looking difference equation + shock(s)
\item Agents observe everything
\item Enough conditions to deliver unique stationary solution
\end{itemize}
\item Question:  Can we identify parameters when we observe \\
\begin{itemize}
\item all the shocks?
\item none of the shocks?
\item some of the shocks?
\item forecasts or asset prices?
\end{itemize}
\end{itemize}
\end{frame}
\end{comment}
%\end{document}

\section{Example 1: independent shocks}
% --------------------------------------------------------------------------
\begin{frame}
\frametitle{Example 1:  independent shocks}
\begin{itemize} \itemsep=\bigskipamount
\item Independent moving average shocks
\item Vary what we observe
\item Apply to model with Taylor rule
\end{itemize}
\end{frame}
%\end{document}

% --------------------------------------------------------------------------
\begin{frame}
\frametitle{Example 1:  independent shocks}
\begin{itemize} \itemsep=\bigskipamount
\item Model
\begin{eqnarray*}
    y_t &=& \lambda E_t y_{t+1} + x_t \phantom{xxx} \\
    x_t &=& \alpha(L) w_t, \;\;\;   w_t \sim \mbox{NID}(0,1)  %\phantom{xx}
\end{eqnarray*}
[$|\lambda|<1$, $\alpha$ one-sided and square-summable, $w$ fundamental for $x$]
\item Solution (agents observe everything)
\begin{eqnarray*}
    y_t &=& \beta(L) w_t, \;\;\;
    \beta_j (\alpha,\lambda) \;=\; \alpha_j + \lambda \alpha_{j+1} + \lambda^2 \alpha_{j+2} \; \cdots
%      \phantom{xxxxxxxx}
\end{eqnarray*}
%\item Example: If $x$ is MA(1), then
%$   \beta_0 = \alpha_0 + \lambda \alpha_{1} $, $ \beta_1 = \alpha_1 $

\item What do we need to estimate $\lambda$?
%  Notes:
%   * can't run regression, x drives y, not error term
%   * could use forecast, but still need to know x
\end{itemize}
\end{frame}
%\end{document}

% --------------------------------------------------------------------------
\begin{frame}
\frametitle{Observable shock}
\begin{itemize}  \itemsep=\bigskipamount
\item $(x,y)$ \\
\begin{itemize}
\item Joint process but degenerate (same innovations)
\item Cross-equation restrictions on parameters ($\beta$ depends on $\alpha, \lambda$)
\end{itemize}
\item Observe variable $y$ (always) \\
\begin{itemize}
\item Estimate $\beta$
\end{itemize}
\item Observe shock $x$ \\
\begin{itemize}
\item Estimate $\alpha$
\item Use mapping $\beta(\alpha,\lambda)$ to infer $\lambda$
\item $\lambda$ overdetermined if $x$ is MA($q$) and $q>1$
\end{itemize}
\end{itemize}
\end{frame}
%\end{document}

% --------------------------------------------------------------------------
\begin{frame}
\frametitle{Hidden shock}
\begin{itemize}  \itemsep=\bigskipamount
\item $(x,y)$ still joint process, but we don't observe $x$
\item Observe variable $y$ \\
\begin{itemize}
\item Estimate $\beta$
\item But without $x$, we can't distinguish roles of $\alpha$ and $\lambda$
\end{itemize}
\item Do forecasts help?
\begin{eqnarray*}
    f^k_t \;\;=\;\; E_t y_{t+k} &=& \left[ \beta(L)/L^k\right]_+ w_t  \phantom{xxxx}
\end{eqnarray*}
\hspace{0.4in}[Nope, depends only on $\beta$, can't tell us about $(\lambda,\alpha)$]
%\begin{itemize}
%\item[] Adds no information beyond $y$, which we already know
%\end{itemize}
\end{itemize}
\end{frame}
%\end{document}

% --------------------------------------------------------------------------
\begin{frame}
\frametitle{Partly observable shock}
\begin{itemize}  \itemsep=\bigskipamount
\item Model and solution
\begin{eqnarray*}
    y_t &=& \lambda E_t y_{t+1} + x_{1t} + x_{2t} \\
    x_{it} &=& \alpha_{i}(L) w_{it}, \;\;\; (w_{1t},w_{2t}) \sim \mbox{NID}(0,I)
                \phantom{xxxxxx} \\
    \Rightarrow \;\;\;
    y_t &=&  \beta_1(L) w_{1t} + \beta_2(L) w_{2t}
\end{eqnarray*}
\item Observe variable $y$ and shock $x_1$ --- but not $x_2$ \\
\begin{itemize}
\item Estimate $\alpha_{1}$ and $\beta_{1}$
\item Use mapping $\beta_1(\alpha_1,\lambda)$ to infer $\lambda$
\end{itemize}
\item Do forecasts help?  \\
\begin{itemize}
\item They help us identify the $w_i$'s
\item Then we can recover $x_2$ from the difference equation
\end{itemize}
\end{itemize}
\end{frame}
%\end{document}

% --------------------------------------------------------------------------
\begin{frame}
\frametitle{Model with Taylor rule}
\begin{itemize} \itemsep=\bigskipamount
\item Model ($p$ = inflation, $i$ = nominal interest rate)
\begin{align*}
    i_t &\;\;=\;\; r + E_t p_{t+1} + x_{1t} \tag{Euler equation} \\
    i_t &\;\;=\;\; r + \tau p_{t} +  x_{2t} \tag{Taylor rule} \\
    x_{it} &\;\;=\;\; \alpha_i(L) w_{it}, \;\;\; (w_{1t},w_{2t}) \sim \mbox{NID}(0,I)
            \phantom{xx}
\end{align*}
\item Solution
\begin{eqnarray*}
%    p_t &=&  \tau^{-1} E_t p_{t+1} + \tau^{-1} (x_{1t} - x_{2t})
%        \phantom{\sum} \\
%    \phantom{x}
%    \Rightarrow \;\;\;
    \phantom{xxxx}
    p_t &=& \beta_1(L) w_{1t} + \beta_2(L) w_{2t} \\
    i_t &=& r + [\beta_1(L)/L]_+ w_{1t} + [\beta_2(L)/L]_+ w_{2t} + x_{1t}
             \phantom{xx}
\end{eqnarray*}
\item What do we need to estimate $\lambda = \tau^{-1}$?
\end{itemize}
\end{frame}
%\end{document}

% --------------------------------------------------------------------------
\begin{frame}
\frametitle{Taylor rule:  observable shocks}
\begin{itemize}  \itemsep=\bigskipamount
\item $(x_1,x_2,p,i)$ \\
\begin{itemize}
\item Joint process but degenerate (two-dimensional innovation)
\item Cross-equation restrictions on parameters
\end{itemize}
\item Observe shocks $(x_1,x_2)$ \\
\begin{itemize}
\item Estimate $\alpha_i$'s
\end{itemize}
\item Observe variables $(p,i)$ \\
\begin{itemize}
\item Estimate $\beta_i$'s
\item Use mapping $\beta_i(\alpha_i,\lambda)$ to infer $\lambda = \tau^{-1}$
%\item Generally overdetermined
\end{itemize}
\end{itemize}
\end{frame}
%\end{document}

% --------------------------------------------------------------------------
\begin{frame}
\frametitle{Taylor rule:  hidden shock (``Cochrane version'')}
\begin{itemize}  \itemsep=\bigskipamount
\item Shocks:  $x_1 = 0$, $x_2$ hidden
\item Observe variable $p$  \\
\begin{itemize}
\item Estimate $\beta_2$  (possible because shock is one dimensional)
\item Can't distinguish roles of $\alpha_2$ and $\lambda$
\end{itemize}
\item Does extra variable $i$ help?  \\
\begin{itemize}
\item No, contains no information not in $p$
\end{itemize}
\item Do forecasts help? \\
\begin{itemize}
\item No, same reason
\end{itemize}
\end{itemize}
\end{frame}
%\end{document}

% --------------------------------------------------------------------------
\begin{frame}
\frametitle{Taylor rule:  partly observable shock (``Gertler version'')}
\begin{itemize}  \itemsep=\bigskipamount
\item Observe variables $(p,i)$ and shock $x_1$ \\
\begin{itemize}
\item Estimate $\alpha_{1}$ and $\beta_{1}$
\item Use mapping  $\beta_1(\alpha_1,\lambda)$ to infer $\lambda$
\end{itemize}
\item Key insight \\
\begin{itemize}
\item $i$ gives us an additional observable when $x_1 \neq 0$
%\item Allows us to identify $\lambda$
\end{itemize}
\item How might we observe $x_1$?  \\
\begin{itemize}
\item Observe directly
\item Or observe $E_t p_{t+1}$ and back out of Euler equation
%\item ?? make sure this lines up with later stuff
\end{itemize}
\end{itemize}
\end{frame}
%\end{document}

% --------------------------------------------------------------------------
\begin{frame}
\frametitle{Example 1:  summary}
\begin{itemize} \itemsep=\bigskipamount
\item Identification possible with partial observability
\item Observability facilitated by forecasts
\item Independence needs further investigation
\item For later:  Where does the Euler equation shock come from?
\end{itemize}
\end{frame}
%\end{document}

\section{Example 2: correlated shocks}
% --------------------------------------------------------------------------
\begin{frame}
\frametitle{Example 2:  correlated shocks}
\begin{itemize} \itemsep=\bigskipamount
\item Correlated shocks in state-space model
\item Vary what we observe
\item Apply to model with Taylor rule
\end{itemize}
\end{frame}
%\end{document}

% --------------------------------------------------------------------------
\begin{frame}
\frametitle{Example 2:  correlated shocks}
\begin{itemize} \itemsep=\bigskipamount
\item Model
\begin{eqnarray*}
    y_t &=& \lambda E_t y_{t+1} + e^\top x_t  \\
    x_{t+1} &=& A x_{t} + C w_{t+1},
            \;\;\;\{ w_t \} \sim \mbox{NID}(0,I)
\end{eqnarray*}
[$|\lambda|<1$, $A$ stable \& ``regular'']
\item Solution %    (agents observe everything)
\begin{eqnarray*}
    y_t &=&  e^\top (I-\lambda A)^{-1} x_t \;\;=\;\; \beta^\top x_t
        \phantom{xxx}
\end{eqnarray*}
%\smallskip
\item What do we need to estimate $\lambda$? %\\
%\begin{itemize}
%\item Two observability issues:  state $x$ and shock $e^\top x$
%\end{itemize}
\end{itemize}
\end{frame}
%\end{document}

% --------------------------------------------------------------------------
\begin{frame}
\frametitle{Observable state and shock}
\begin{itemize}  \itemsep=\bigskipamount
\item Observe state $x$ and vector $e$ (hence shock $e^\top x$)\\
\begin{itemize}
\item Estimate $A$ %\;\; [$C$, too, but it's not needed here]
\end{itemize}
\item Observe variable $y$ \\
\begin{itemize}
\item Estimate $\beta$, use $A$ to infer $\lambda$
\begin{eqnarray*}
    \lambda \beta^\top &=& (\beta - e)^\top A^{-1}
    \phantom{xxxxxx}
     %\;\;\; (\lambda \mbox { is overdetermined)}
\end{eqnarray*}
\item[] [$\lambda$ typically overdetermined]
\end{itemize}
\item Forecasts redundant
\begin{eqnarray*}
    f^k_t \;\;=\;\; E_t y_{t+k} &=& \beta^\top A^k x_t \phantom{xxxxx}
\end{eqnarray*}
\end{itemize}
\end{frame}
%\end{document}

% --------------------------------------------------------------------------
\begin{frame}
\frametitle{Digression on hidden shock \& state}
\begin{itemize}  \itemsep=\bigskipamount
\item What's hidden, state $x$ or shock $e^\top x$?
\item One line of thought (Hansen-Sargent) \\
\begin{itemize}
\item Only part of state observed
\item $\Rightarrow$ can't estimate $\beta$, innovations may not be fundamental
\end{itemize}
\item We suggest:  observe the state, not the shock  \\
\begin{itemize} \itemsep=0.25\bigskipamount
\item Lots of econometric approaches  \\
\item Or:  build it from forecasts of $y$
\begin{eqnarray*}
    f_t &=& [ y_t, f^1_t, \cdots, f^{n-1}_t ]^\top
            \phantom{xxxxxx}
\end{eqnarray*}
$f_t$ spans state $x$ if $A$ is regular and $n = \dim(x)$
%\item This leads to transformed state-space equations
%\begin{eqnarray*}
%    f_t &=&  F x_t  \\
%    \;\;\; \Rightarrow
%    f_{t+1} &=& F^{-1} A F f_t + F^{-1} C w_{t+1}
%        \;\;=\;\; \widehat{A} f_{t} + \widehat{C} w_{t+1} \\
%    y_t &=& e^\top F^{-1} (I-\lambda \widehat{A})^{-1} f_t
%            \;\;=\;\;
%            \widehat{e}^\top (I-\lambda \widehat{A})^{-1} f_t \;\;=\;\; %\widehat{\beta}^{\top} f_t
%            \phantom{xxxxxxxxxx}
%\end{eqnarray*}
\end{itemize}
%\item From here on:  assume we observe $x$
\end{itemize}
\end{frame}
%\end{document}

% --------------------------------------------------------------------------
\begin{frame}
\frametitle{Hidden shock}
\begin{itemize}  \itemsep=\bigskipamount
\item Shock $e^\top x$ not observed ($e$ not known)
\item Observe variable $y$, state $x$ \\
\begin{itemize}
\item From $y$:  estimate $\beta$
\item From $x$:  estimate $A$
\item But since we don't know $e$, can't infer $\lambda$
\end{itemize}
\item Do forecasts help? \\
\begin{itemize}
\item They span the state, not otherwise helpful
\end{itemize}
\end{itemize}
\end{frame}
%\end{document}

% --------------------------------------------------------------------------
\begin{frame}
\frametitle{Partly observable shock}
\begin{itemize}  \itemsep=\bigskipamount
\item Model
\begin{eqnarray*}
    y_t &=& \lambda E_t y_{t+1} + e_1^\top x_{t} + e_2^\top x_{t}
    \\
   \Rightarrow  \;\;\;
    y_t &=&  (e_1+e_2)^\top (I-\lambda A)^{-1} x_t \;\;=\;\; \beta^\top x_t
    \phantom{xxxxx}
\end{eqnarray*}
\item Observe variable $y$, state $x$ \\
\begin{itemize}
\item From $y$:  estimate $\beta$
\item From $x$:  estimate $A$
\end{itemize}
\item Is it enough to observe shock $e_1^\top x$?  \\
\begin{itemize}
\item Can't easily disentangle effects of $e_1$ and $e_2$ \\
\end{itemize}
\end{itemize}
\end{frame}
%\end{document}

% --------------------------------------------------------------------------
\begin{frame}
\frametitle{Partly observable shock (continued)}
\begin{itemize}  \itemsep=\bigskipamount
\item Reminder:  solution is
\begin{eqnarray*}
     \beta^\top  &=&  (e_1+e_2)^\top (I-\lambda A)^{-1}
    \phantom{xxxxx}
\end{eqnarray*}
\item What we know:  $\beta$ ($n$ knowns)
\item What we don't know:  $e_1, \lambda$ ($n+1$ unknowns)
\item Needed:  one or more restrictions \\
\begin{itemize}
\item Uncorrelated shocks
\item Zero restrictions on $e_2$
\item Tight economic structure
\end{itemize}
\end{itemize}
\end{frame}
%\end{document}

% --------------------------------------------------------------------------
\begin{frame}
\frametitle{Model with Taylor rule}
\begin{itemize}  \itemsep=\bigskipamount
\item Model ($p$ = inflation, $i$ = nominal interest rate)
\begin{align*}
    \phantom{xxxxxxxx}
    i_t &\;\;=\;\;  r +  E_t p_{t+1} + e_1^\top x_t  \tag{Euler equation} \\
    i_t &\;\;=\;\;  r + \tau p_{t} +  e_2^\top x_{t} \tag{Taylor rule} \\
    x_{t+1} &\;\;=\;\; A x_{t} + C w_{t+1},
            \;\;\;\{ w_t \} \sim \mbox{NID}(0,I)
\end{align*}
\item Solution:  $ p_t = \beta^\top x_t $
\begin{eqnarray*}
    \beta^\top &=& (e_1 - e_2)^\top (\tau I - A)^{-1}
    \phantom{xxxxx}
\end{eqnarray*}
\item What do we need to estimate $\lambda = \tau^{-1}$?  \\
\begin{itemize}
\item Same structure as example
\item One restriction on $e_2$ is enough
\end{itemize}
\end{itemize}
\end{frame}
%\end{document}

% --------------------------------------------------------------------------
\begin{frame}
\frametitle{Example 2:  summary}
\begin{itemize} \itemsep=\bigskipamount
\item Forecasts helpful in observing state
\item Identification then possible with \\
\begin{itemize}
\item Partial observability of shocks
\item Restriction(s) on hidden (monetary policy) shock
\end{itemize}
\item Identifying parameters and shocks are related
\item Next up:  Where does the Euler equation shock come from?
\end{itemize}
\end{frame}
%\end{document}

\section{Example 3: bond pricing models}
% --------------------------------------------------------------------------
\begin{frame}
\frametitle{Example 3:  bond pricing models}
\begin{itemize} \itemsep=\bigskipamount
\item Shocks to Euler equation follow naturally
\item Additional restrictions between means and variances
\item Representative agent (now) and ``affine'' examples (later)
\end{itemize}
\end{frame}
%\end{document}

% --------------------------------------------------------------------------
\begin{frame}
\frametitle{Representative agent model}
\begin{itemize} \itemsep=\bigskipamount
\item Model
\begin{align*}
    i_t &\;\;=\;\;  - \log E_t \left[ m_{t+1} \exp(-p_{t+1}) \right]
                \tag{Euler equation} \\
    i_t &\;\;=\;\;  r + \tau p_{t} +  e_2^\top x_{t} \tag{Taylor rule} \\
    \log m_{t+1} &\;\;=\;\; - \delta - \alpha \log g_{t+1}  \\
    \log g_{t+1} &\;\;=\;\; g + e_1^\top x_t \\
    x_{t+1} &\;\;=\;\; A x_{t} + C w_{t+1},
            \;\;\;\{ w_t \} \sim \mbox{NID}(0,I)
\end{align*}

\item Solution:  $ p_t = \beta^\top x_t $
\begin{eqnarray*}
    \beta^\top &=& (\alpha e_1 - e_2)^\top (\tau I - A)^{-1} \\
    r &=& \delta + \alpha g - (\alpha e_1 + \beta)^\top C C^\top (\alpha e_1 + \beta)/2
\end{eqnarray*}
\end{itemize}
\end{frame}
%\end{document}

% --------------------------------------------------------------------------
\begin{frame}
\frametitle{Representative agent model: identification}
\begin{itemize} \itemsep=\bigskipamount
\item What do we need to estimate $\lambda = \tau^{-1}$?
\item Reminder:   solution is
\begin{eqnarray*}
    \beta^\top &=& (\alpha e_1 - e_2)^\top (\tau I - A)^{-1} \\
    r &=& \delta + \alpha g - (\alpha e_1 + \beta)^\top C C^\top (\alpha e_1 + \beta)/2
\end{eqnarray*}
\item Identification strategies \\
\begin{itemize}
\item Observe $\log g_{t}$, $e_1$
\item Long bond yields span state $x$ (they're like forecasts here)
\item Risk aversion $\alpha$:  match equity premium
\item One restriction on $e_2$
\item {\bf NEW:} Additional information from mean bond yields
\end{itemize}
\end{itemize}
\end{frame}
%\end{document}

\begin{comment}
% --------------------------------------------------------------------------
\begin{frame}
\frametitle{Example 3:  summary}
\begin{itemize} \itemsep=\bigskipamount
\item Shocks to Euler equation follow naturally
\item Additional restrictions between means and variances
\item Representative agent (now) and ``affine'' examples (later)
\end{itemize}
\end{frame}
\end{comment}
%\end{document}

\section{ }
% --------------------------------------------------------------------------
\begin{frame}
\frametitle{Summary and conclusions}

\begin{itemize} \itemsep=\bigskipamount
\item Identification never a free lunch
\item We suggest \\
\begin{itemize}
\item Forecasts and bond yields allow us to observe the state
\item Additional restrictions can identify shock and ``Taylor rule''
\item Common bond pricing models generate such restrictions
\end{itemize}
\item The biggest challenge is not identification but fit
\item Open questions \\
\begin{itemize}
\item What kinds of models do we need to approximate the data?
\item Do bond yields span the state?
\end{itemize}
\end{itemize}
\end{frame}

\section{ }
% --------------------------------------------------------------------------
\begin{frame}
\frametitle{Related work (some of it)}

\begin{itemize} \itemsep=0.5\bigskipamount
\item Identifying monetary policy shocks \\
\begin{itemize}
\item Christiano-Eichenbaum-Evans,
Clarida-Gali-Gertler,
Cochrane, Hansen-Sargent,
Leeper-Sims-Zha
\end{itemize}

\item Factor models \\
\begin{itemize}
\item Sargent-Sims, Stock-Watson
\end{itemize}

\item Macro bond pricing models  \\
\begin{itemize}
\item Gallmeyer-Hollifield(-Palomino)-Zin,
Gurkaynak-Sack-Swanson,
Piazzesi-Schneider,
Rudebusch-Swanson,
Smith-Taylor
\end{itemize}

\item Markov property of bond yields   \\
\begin{itemize}
\item Cochrane-Piazzesi, Collin-Dufresne-Goldstein
\end{itemize}

\end{itemize}
\end{frame}


\end{document}

% --------------------------------------------------------------------------
% --------------------------------------------------------------------------
NYU, March 2012

This paper started with a conversation Stan and I had with Mark several years ago,
but I'll set that aside for a moment.

The issue is how you identify the Taylor rule: if the Fed changes its
policy rule, how would you see that in the data?
The issue, as commonly addressed, is one of how much we observe.
If we don't observe the shock to the monetary policy rule,
can we still identify the response of the interest rate to inflation?

We're going


Mike, Stan, and I :
eg, equity premium.
Another is the term structure.


% --------------------------------------------------------------------------
Minnesota, May 2012
This is an old school paper, which I like,
but it's also an attempt to make up for missing out on Minn in the 1970s.

When we were graduate students at Yale,
Tim and I were sitting around reading Samuelson's Foundations...

Now what we didn't realize at the time was that the 1970s was a great
time to study economics --- at Minnesota.
With Patrick's help...  We read papers by Sargent, Wallce, Sims, and their students
and tried to get up to speed.
A little later, we met the authors.  And we said to ourselves, that's it?

This paper...
The idea is to use

We build on work by Lars, Tom, and Chris.
Mike's guess is that Lars will tell us he knew this in 1979,
and he probably did,
that Chris will say it's wrong,
and that Tom will agree with both of them.

